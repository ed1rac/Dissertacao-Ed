% --- -----------------------------------------------------------------
% --- Elementos usados na Capa e na Folha de Rosto.
% --- EXPRESSÔES ENTRE <> DEVERÂO SER COMPLETADAS COM A INFORMAÇÂO ESPECÍFICA DO TRABALHO
% --- E OS SÌMBOLOS <> DEVEM SER RETIRADOS 
% --- -----------------------------------------------------------------
\autor{EDKALLENN SILVA DE LIMA} % deve ser escrito em maiusculo

\titulo{Aplicação de Mineração de dados e Aprendizagem de Máquina na detecção de conflitos entre políticas}

\instituicao{UNIVERSIDADE FEDERAL DO ACRE}

\orientador{DRA. LAURA COSTA SARKIS}

\local{RIO BRANCO - ACRE}

\data{2020} % ano da defesa

\comentario{Dissertação de Mestrado apresentada ao Programa de P\'{o}s-Gradua\c{c}\~{a}o em Computa\c{c}\~{a}o da \mbox{Universidade} Federal do Acre como requisito parcial para a obten\c{c}\~{a}o do Grau de \mbox{Mestre em Ciência da Computação}. \'{A}rea de concentra\c{c}\~{a}o: \mbox{Engenharia de Sistemas de Informação}} %preencha com a sua area de concentracao


% --- -----------------------------------------------------------------
% --- Capa. (Capa externa, aquela com as letrinhas douradas)(Obrigatorio)
% --- ----------------------------------------------------------------
\begin{figure}
	\centering
	\includegraphics[width=.25\textwidth]{imagens/brasao_UFAC.png}	
	\label{fig:UFAC}
\end{figure}
\capa

% --- -----------------------------------------------------------------
% --- Folha de rosto. (Obrigatorio)
% --- ----------------------------------------------------------------

\pagestyle{ruledheader}
\setcounter{page}{1}
\pagenumbering{roman}

% --- -----------------------------------------------------------------
% --- Termo de aprovacao. (Obrigatorio)
% --- ----------------------------------------------------------------
\cleardoublepage
\thispagestyle{empty}

\vspace{-60mm}

\begin{center}
   {\large EDKALLENN SILVA DE LIMA}\\
   \vspace{7mm}

  Aplicação de Mineração de dados e Aprendizagem de Máquina na detecção de conflitos entre políticas\\
  \vspace{10mm}
\end{center}

\noindent
\begin{flushright}
\begin{minipage}[t]{8cm}

Dissertação de Mestrado apresentada ao Programa de Pós-Gradua\c{c}\~{a}o em Ciência da Computação da Universidade Federal do Acre como requisito parcial para a obtenção do Grau de Mestre em Ciência da Computação. \'{A}rea de concentra\c{c}\~{a}o: \mbox{Engenharia de Sistemas de Informação} 
%preencha com a sua area de concentracao

\end{minipage}
\end{flushright}
\vspace{1.0 cm}
\noindent
{Aprovada em <MES> de 2020.} \\
\begin{flushright}
  \parbox{11cm}
  {
  \begin{center}
  BANCA EXAMINADORA \\
  \vspace{6mm}
  \rule{11cm}{.1mm} \\
    Prof. DRA. LAURA COSTA SARKIS - Orientador, UFAC \\
    \vspace{6mm}
  \rule{11cm}{.1mm} \\
    Prof. <NOME DO AVALIADOR>, <INSTITUI\c{C}\~AO>\\
    \vspace{6mm}
  \rule{11cm}{.1mm} \\
    Prof. <NOME DO AVALIADOR>, <INSTITUI\c{C}\~AO>\\
  \vspace{4mm}
  \rule{11cm}{.1mm} \\
    Prof. <NOME DO AVALIADOR>, <INSTITUI\c{C}\~AO>\\
    \vspace{6mm}

  \end{center}
  }
\end{flushright}
\begin{center}
  \vspace{4mm}
  RIO BRANCO - ACRE \\
  %\vspace{6mm}
  2020

\end{center}

% --- -----------------------------------------------------------------
% --- Dedicatoria.(Opcional)
% --- -----------------------------------------------------------------
\cleardoublepage
\thispagestyle{empty}
\vspace*{200mm}

\begin{flushright}
{\em 
Dedicat\'oria(s): Este trabalho não seria possível sem a educação que me foi concedida, os ensinamentos e a sabedoria oriundos de você, Lucimar do Rego Albuquerque de Lima, muito mais que uma mãe, uma inspiração. <IN MEMORIAN>
}
\end{flushright}
\newpage


% --- -----------------------------------------------------------------
% --- Citação.(Opcional)
% --- -----------------------------------------------------------------
\cleardoublepage
\thispagestyle{empty}
\vspace*{200mm}

\begin{flushright}
	{\em 
		“Assim como casas são feitas de pedras, a ciência é feita de fatos. Mas uma pilha de pedras não é uma casa e uma coleção de fatos não é, necessariamente, ciência”.  \\ \textbf{Jules Henri Poincaré}, matemático, físico e filósofo da ciência francês
	}
\end{flushright}
\newpage




% --- -----------------------------------------------------------------
% --- Agradecimentos.(Opcional)
% --- -----------------------------------------------------------------
\pretextualchapter{Agradecimentos}
\hspace{5mm}
Lorem ipsum dolor sit amet, consectetur adipiscing elit. Duis rutrum maximus fermentum. Duis id quam nibh. Aliquam cursus eget mauris at fermentum. Nam iaculis posuere sapien. Praesent facilisis nisl ipsum, at gravida est scelerisque non. In sed luctus sem. Integer odio sem, feugiat eu eros eget, lacinia lobortis mi. Donec dapibus, tellus non consequat aliquet, dui massa suscipit dolor, nec pretium tortor massa vitae lorem. Quisque non faucibus tellus. Quisque dignissim nibh quis sem imperdiet bibendum. Duis nec dictum turpis. Vestibulum auctor leo nulla, sed venenatis purus placerat et. Nam euismod mauris in justo semper laoreet.

Cras rutrum faucibus eros et feugiat. In sit amet viverra lorem. Sed euismod purus eget sodales ultricies. Praesent et blandit lorem. Pellentesque ut tempus velit. Aenean et vulputate arcu. Suspendisse finibus, tellus eu sollicitudin rutrum, augue nisi maximus lacus, ut blandit ante mi ut nulla.

Aenean metus dui, rhoncus sit amet pretium ut, laoreet quis ipsum. Sed non tempus turpis, ac eleifend lacus. Integer vel dui finibus, imperdiet neque porttitor, venenatis nunc. Vivamus egestas, turpis quis porttitor tincidunt, purus risus suscipit felis, nec consequat justo quam at purus. Pellentesque consequat sem ut nibh malesuada pulvinar. Quisque lobortis pulvinar urna, vitae luctus nulla ultricies vestibulum. Sed accumsan tortor tellus, in aliquam tellus sollicitudin eu. Vestibulum ante ipsum primis in faucibus orci luctus et ultrices posuere cubilia curae; Duis justo ipsum, vestibulum at magna id, ultrices lobortis turpis. Vivamus porttitor scelerisque odio sit amet posuere. Morbi fermentum ultricies sem, dictum pulvinar turpis. 

% --- -----------------------------------------------------------------
% --- Resumo em portugues.(Obrigatorio)
% --- -----------------------------------------------------------------
\begin{resumo}


A quantidade de informações disponíveis cresce a cada ano. Aumenta, junto com o volume de dados e informações, o interesse em tratar, analisar e descobrir conhecimento a partir desta avalanche de dados. A mineração de dados juntamente com o aprendizado de máquina são duas ferramentas-chave dentro deste processo de descoberta de conhecimento e utilização de todos esses dados e informações para propósitos úteis. Entretanto, antes que os dados possam ser analisados, eles precisam ser armazenados e os sistemas computacionais sofrem, também de forma crescente, permanentes ameaças à sua segurança. Neste contexto se inserem as políticas para sistemas computacionais que buscam garantir meios para proteção, confidencialidade e confiabilidade dos acessos dos usuários aos objetos dentro dos sistemas de uma organização. Em sistemas com múltiplos sujeitos, muitas ações e diversos objetos, eventualmente, ocorrerão conflitos entre políticas. Um conflito ocorre quando os objetivos de duas ou mais políticas não podem ser atendidos simultaneamente em determinado contexto. Este trabalho tem como objetivo propor que o problema da detecção de conflitos em políticas pode ser convertido em um problema de \textit{data mining} (mineração de dados) resolvido pela tarefa da classificação além de modelar e sintetizar uma forma de detectar estes conflitos mediante o uso de diferentes algoritmos e técnicas da aprendizagem de máquina que consigam acurácias elevadas e forneçam modelos genéricos o suficiente para serem usados em outros contextos.

{\hspace{-8mm} \bf{Palavras-chave}}: Controle de Acesso. Mineração de dados. Aprendizagem de máquina. Conflitos diretos. Conflitos indiretos. Detecção de conflitos.

\end{resumo}

% --- -----------------------------------------------------------------
% --- Resumo em lingua estrangeira.(Obrigatorio)
% --- -----------------------------------------------------------------
\begin{abstract}

The amount of information available grows every year. It increases, along with the volume of data and information, the interest in treating, analyzing and discovering knowledge from this avalanche of data. Data mining associated with machine learning are two key tools within this process of discovering knowledge and using all that data and information for useful purposes. However, before data can be analyzed, it needs to be stored and computer systems are also increasingly threatened with permanent security threats. In this context, policies for computer systems are inserted as a way to guarantee protection, confidentiality and reliability of users' access to objects within an organization's systems. In systems with multiple subjects, many actions and different objects, eventually, conflicts between policies will occur. A conflict occurs when the objectives of two or more policies cannot be met simultaneously in a given context. This work aims to propose that the problem of detecting conflicts in policies can be converted into a problem of data mining (data mining) solved by the task of classification, in addition to modeling and synthesizing a way of detecting these conflicts through the use of different machine learning algorithms and techniques that achieve high accuracy and provide generic models to be used in other contexts.


{\hspace{-8mm} \bf{Keywords}}: Access control. Data mining. Machine learning. Direct conflicts. Indirect conflicts. Conflict detection.

\end{abstract}

% --- -----------------------------------------------------------------
% --- Lista de figuras.(Opcional)
% --- -----------------------------------------------------------------
%\cleardoublepage
\listoffigures


% --- -----------------------------------------------------------------
% --- Lista de tabelas.(Opcional)
% --- -----------------------------------------------------------------
\cleardoublepage
%\label{pag:last_page_introduction}
\listoftables
\cleardoublepage

% --- -----------------------------------------------------------------
% --- Lista de abreviatura.(Opcional)
%Elemento opcional, que consiste na relação alfabética das abreviaturas e siglas utilizadas no texto, seguidas das %palavras ou expressões correspondentes grafadas por extenso. Recomenda-se a elaboração de lista própria para cada %tipo (ABNT, 2005).
% --- ----------------------------------------------------------------
\cleardoublepage
\pretextualchapter{Lista de Abreviaturas e Siglas}
\begin{tabular}{lcl}	
AM & : & \textit{Aprendizado de Máquina};\\
ATM & : & \textit{Air-Traffic Management};\\
BPNN & : & \textit{Basic Probabilistic Neural Network};\\
CPNN & : & \textit{Constructive Probabilistic Neural Network};\\
DAC & : & \textit{Discretionary Access Control};\\
DARPA & : & \textit{Defense Advanced Research Projects Agency};\\
DoS & : & \textit{Denial of Service};\\
GPGU & : & \textit{General Purpose Graphic Processor Unit};\\
GPU & : & \textit{Graphic Processor Unit};\\
KDD & : & \textit{Knowledege Discovery in Data Bases};\\
MAC & : & \textit{Mandatory Access Control};\\
ML & : & \textit{Machine Learning};\\
MLP & : & \textit{MultiLayer Perceptron};\\
NFL & : & \textit{No-Free Lucnch};\\
PMC & : & \textit{Perceptron MultiCamadas};\\
RBAC & : & \textit{Role-Based Access Control};\\
ReBAC & : & \textit{Relationship-Based Access Control};\\
RNA & : & \textit{Rede Neural Artificial};\\
RTLS & : & \textit{Real Time Location Systems};\\
SMA & : & \textit{Sistema Multi Agentes};\\
SVM & : & \textit{Support Vector Machiner};\\
\end{tabular}
% --- -----------------------------------------------------------------
% --- Sumario.(Obrigatorio)
% --- -----------------------------------------------------------------
\pagestyle{ruledheader}
\tableofcontents


