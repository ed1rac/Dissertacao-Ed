
\documentclass[
	% -- opções da classe memoir --
	12pt,				% tamanho da fonte
	openright,			% capítulos começam em pág ímpar (insere página vazia caso preciso)
	oneside,			% twoside para impressão em recto e verso. Oposto a oneside
	a4paper,			% tamanho do papel. 
	% -- opções da classe abntex2 --
	%chapter=TITLE,		% títulos de capítulos convertidos em letras maiúsculas
	%section=TITLE,		% títulos de seções convertidos em letras maiúsculas
	%subsection=TITLE,	% títulos de subseções convertidos em letras maiúsculas
	%subsubsection=TITLE,% títulos de subsubseções convertidos em letras maiúsculas
	% -- opções do pacote babel --
	english,			% idioma adicional para hifenização
	french,				% idioma adicional para hifenização
	spanish,			% idioma adicional para hifenização
	brazil				% o último idioma é o principal do documento
	]{abntex2}

% ---
% Pacotes básicos 
% ---
\usepackage{lmodern}			% Usa a fonte Latin Modern			
\usepackage[T1]{fontenc}		% Selecao de codigos de fonte.
\usepackage[utf8]{inputenc}		% Codificacao do documento (conversão automática dos acentos)
\usepackage{indentfirst}		% Indenta o primeiro parágrafo de cada seção.
\usepackage{color}				% Controle das cores
\usepackage{graphicx}			% Inclusão de gráficos
\usepackage{microtype} 			% para melhorias de justificação
% ---
	
%\usepackage[brazil]{babel}		
\usepackage{amsmath}
% ---
% Pacotes adicionais, usados apenas no âmbito do Modelo Canônico do abnteX2
% ---
\usepackage{lipsum}				% para geração de dummy text
% ---

% ---
% Pacotes de citações
% ---
\usepackage[brazilian,hyperpageref]{backref}	 % Paginas com as citações na bibl
\usepackage[alf]{abntex2cite}	% Citações padrão ABNT

% ---
% Outros pacotes
% ---
%--- pacote para gerar pseudo-codigo
\usepackage{algorithm}
\usepackage{algorithmic}
%\userpackage{float}
\floatname{algorithm}{Algoritmo}
\usepackage{listings}
%Tabela Colorida
\usepackage{colortbl}
%Outros
\usepackage{multicol}
\usepackage{multirow}
\usepackage{rotating}


\hyphenation{
	a-de-qua-da-men-te 
	di-men-sio-na-men-to 
}
% --- 
% CONFIGURAÇÕES DE PACOTES
% --- 

% ---
% Configurações do pacote backref
% Usado sem a opção hyperpageref de backref
\renewcommand{\backrefpagesname}{Citado na(s) página(s):~}
% Texto padrão antes do número das páginas
\renewcommand{\backref}{}
% Define os textos da citação
\renewcommand*{\backrefalt}[4]{
	\ifcase #1 %
		Nenhuma citação no texto.%
	\or
		Citado na página #2.%
	\else
		Citado #1 vezes nas páginas #2.%
	\fi}%
% ---

% ---
% Informações de dados para CAPA e FOLHA DE ROSTO
% ---
\titulo{Aplicação de Mineração de dados e Aprendizagem de Máquina na detecção de conflitos entre políticas}
\autor{EDKALLENN SILVA DE LIMA} % deve ser escrito em maiusculo
\local{RIO BRANCO - ACRE}
\data{2020, v-1.1.2}
\orientador{DRA. LAURA COSTA SARKIS}

\instituicao{%
  UNIVERSIDADE FEDERAL DO ACRE -- UFAC
  \par
  Programa de Pós-Graduação em Ciência da Computação -- PPgCC}
\tipotrabalho{Dissertação (Mestrado)}
% O preambulo deve conter o tipo do trabalho, o objetivo, 
% o nome da instituição e a área de concentração 
\preambulo{Dissertação de Mestrado apresentada ao Programa de P\'{o}s-Gradua\c{c}\~{a}o em Computa\c{c}\~{a}o da \mbox{Universidade} Federal do Acre como requisito parcial para a obten\c{c}\~{a}o do Grau de \mbox{Mestre em Ciência da Computação}. \'{A}rea de concentra\c{c}\~{a}o: \mbox{Engenharia de Sistemas de Informação}} %preencha com a sua area de concentracao}
% ---


% ---
% Configurações de aparência do PDF final

% alterando o aspecto da cor azul
\definecolor{blue}{RGB}{41,5,195}

% informações do PDF
\makeatletter
\hypersetup{
     	%pagebackref=true,
		pdftitle={\@title}, 
		pdfauthor={\@author},
    	pdfsubject={\imprimirpreambulo},
	    pdfcreator={LaTeX with abnTeX2},
		pdfkeywords={abnt}{latex}{abntex}{abntex2}{trabalho acadêmico}, 
		colorlinks=true,       		% false: boxed links; true: colored links
    	linkcolor=blue,          	% color of internal links
    	citecolor=blue,        		% color of links to bibliography
    	filecolor=magenta,      	% color of file links
		urlcolor=blue,
		bookmarksdepth=4
}
\makeatother
% --- 

% ---
% Posiciona figuras e tabelas no topo da página quando adicionadas sozinhas
% em um página em branco. Ver https://github.com/abntex/abntex2/issues/170
\makeatletter
\setlength{\@fptop}{5pt} % Set distance from top of page to first float
\makeatother
% ---

% ---
% Possibilita criação de Quadros e Lista de quadros.
% Ver https://github.com/abntex/abntex2/issues/176
%
%\newcommand{\quadroname}{Quadro}
%\newcommand{\listofquadrosname}{Lista de quadros}

%\newfloat[chapter]{quadro}{loq}{\quadroname}
%\newlistof{listofquadros}{loq}{\listofquadrosname}
%\newlistentry{quadro}{loq}{0}

% configurações para atender às regras da ABNT - COMENTEI TUDO
%\setfloatadjustment{quadro}{\centering}
%\counterwithout{quadro}{chapter}
%\renewcommand{\cftquadroname}{\quadroname\space} 
%\renewcommand*{\cftquadroaftersnum}{\hfill--\hfill}

%\setfloatlocations{quadro}{hbtp} % Ver https://github.com/abntex/abntex2/issues/176
% ---

% --- 
% Espaçamentos entre linhas e parágrafos 
% --- 

% O tamanho do parágrafo é dado por:
\setlength{\parindent}{1.3cm}

% Controle do espaçamento entre um parágrafo e outro:
\setlength{\parskip}{0.2cm}  % tente também \onelineskip

% ---
% compila o indice
% ---
\makeindex
% ---

% ----
% Início do documento
% ----
\begin{document}

% Seleciona o idioma do documento (conforme pacotes do babel)
%\selectlanguage{english}
\selectlanguage{brazil}

% Retira espaço extra obsoleto entre as frases.
\frenchspacing 

% ----------------------------------------------------------
% ELEMENTOS PRÉ-TEXTUAIS
% ----------------------------------------------------------
% \pretextual

% ---
% Capa
% ---
% --- -----------------------------------------------------------------
% --- Capa. (Capa externa, aquela com as letrinhas douradas)(Obrigatorio)
% --- ----------------------------------------------------------------
\begin{figure}
	\centering
	\includegraphics[width=.25\textwidth]{imagens/brasao_UFAC.png}	
	\label{fig:UFAC}
\end{figure}
\imprimircapa
% ---

% ---
% Folha de rosto
% (o * indica que haverá a ficha bibliográfica)
% ---
\imprimirfolhaderosto*
% ---
% ---
% Inserir a ficha bibliografica
% ---

% Isto é um exemplo de Ficha Catalográfica, ou ``Dados internacionais de
% catalogação-na-publicação''. Você pode utilizar este modelo como referência. 
% Porém, provavelmente a biblioteca da sua universidade lhe fornecerá um PDF
% com a ficha catalográfica definitiva após a defesa do trabalho. Quando estiver
% com o documento, salve-o como PDF no diretório do seu projeto e substitua todo
% o conteúdo de implementação deste arquivo pelo comando abaixo:
%
% \begin{fichacatalografica}
%     \includepdf{fig_ficha_catalografica.pdf}
% \end{fichacatalografica}

\begin{fichacatalografica}
	\sffamily
	\vspace*{\fill}					% Posição vertical
	\begin{center}					% Minipage Centralizado
	\fbox{\begin{minipage}[c][8cm]{13.5cm}		% Largura
	\small
	\imprimirautor
	%Sobrenome, Nome do autor
	
	\hspace{0.5cm} \imprimirtitulo  / \imprimirautor. --
	\imprimirlocal, \imprimirdata-
	
	\hspace{0.5cm} \thelastpage p. : il. (algumas color.) ; 30 cm.\\
	
	\hspace{0.5cm} \imprimirorientadorRotulo~\imprimirorientador\\
	
	\hspace{0.5cm}
	\parbox[t]{\textwidth}{\imprimirtipotrabalho~--~\imprimirinstituicao,
	\imprimirdata.}\\
	
	\hspace{0.5cm}
		1. Políticas de controle de acesso.
		2. Mineração de Dados.
		2. Aprendizagem de máquina.
		I. DRA. LAURA COSTA SARKIS.
		II. UFAC - Universidade Federal do Acre.
		III.  Programa de Pós-Graduação em Ciência da Computação -- PPgCC.
		IV. Título 			
	\end{minipage}}
	\end{center}
\end{fichacatalografica}
% ---

% ---
% Inserir errata
% ---
%\begin{errata}
%Elemento opcional da \citeonline[4.2.1.2]{NBR14724:2011}. Exemplo:
%
%\vspace{\onelineskip}%

%FERRIGNO, C. R. A. \textbf{Tratamento de neoplasias ósseas apendiculares com
%reimplantação de enxerto ósseo autólogo autoclavado associado ao plasma
%rico em plaquetas}: estudo crítico na cirurgia de preservação de membro em
%cães. 2011. 128 f. Tese (Livre-Docência) - Faculdade de Medicina Veterinária e
%Zootecnia, Universidade de São Paulo, São Paulo, 2011.

%\begin{table}[htb]
%\center
%\footnotesize
%\begin{tabular}{|p{1.4cm}|p{1cm}|p{3cm}|p{3cm}|}
%  \hline
%   \textbf{Folha} & \textbf{Linha}  & \textbf{Onde se lê}  & \textbf{Leia-se}  \\
%    \hline
%    1 & 10 & auto-conclavo & autoconclavo\\
%   \hline
%\end{tabular}
%\end{table}
%
%\end{errata}
% ---

% ---
% Inserir folha de aprovação
% ---

% Isto é um exemplo de Folha de aprovação, elemento obrigatório da NBR
% 14724/2011 (seção 4.2.1.3). Você pode utilizar este modelo até a aprovação
% do trabalho. Após isso, substitua todo o conteúdo deste arquivo por uma
% imagem da página assinada pela banca com o comando abaixo:
%
% \begin{folhadeaprovacao}
% \includepdf{folhadeaprovacao_final.pdf}
% \end{folhadeaprovacao}
%
%\begin{folhadeaprovacao}
%
% \begin{center}
%    {\ABNTEXchapterfont\large\imprimirautor}
%
%    \vspace*{\fill}\vspace*{\fill}
%    \begin{center}
%      \ABNTEXchapterfont\bfseries\Large\imprimirtitulo
%    \end{center}
%    \vspace*{\fill}
%    
%    \hspace{.45\textwidth}
%    \begin{minipage}{.5\textwidth}
%        \imprimirpreambulo
%    \end{minipage}%
%    \vspace*{\fill}
%   \end{center}
%       
%   Trabalho aprovado. \imprimirlocal, xx de xxxxxxxxxxx de 2020:
%
%   \assinatura{\textbf{\imprimirorientador} \\ Orientador} 
%   \assinatura{\textbf{Professor} \\ Convidado 1}
%   \assinatura{\textbf{Professor} \\ Convidado 2}
   %\assinatura{\textbf{Professor} \\ Convidado 3}
   %\assinatura{\textbf{Professor} \\ Convidado 4}
      
%   \begin{center}
%    \vspace*{0.5cm}
%    {\large\imprimirlocal}
%    \par
%    {\large\imprimirdata}
%    \vspace*{1cm}
%  \end{center}
%  
%\end{folhadeaprovacao}
% ---

% --- -----------------------------------------------------------------
% --- Termo de aprovacao. (Obrigatorio)
% --- ----------------------------------------------------------------
\cleardoublepage
\thispagestyle{empty}

\vspace{-60mm}

\begin{center}
	{\large EDKALLENN SILVA DE LIMA}\\
	\vspace{7mm}
	
	Aplicação de Mineração de dados e Aprendizagem de Máquina na detecção de conflitos entre políticas\\
	\vspace{10mm}
\end{center}

\noindent
\begin{flushright}
	\begin{minipage}[t]{8cm}		
		Dissertação de Mestrado apresentada ao Programa de Pós-Gradua\c{c}\~{a}o em Ciência da Computação da Universidade Federal do Acre como requisito parcial para a obtenção do Grau de Mestre em Ciência da Computação. \'{A}rea de concentra\c{c}\~{a}o: \mbox{Engenharia de Sistemas de Informação} 
		%preencha com a sua area de concentracao
		
	\end{minipage}
\end{flushright}
\vspace{1.0 cm}
\noindent
{Aprovada em <MES> de 2020.} \\
\begin{flushright}
	\parbox{11cm}
	{
		\begin{center}
			BANCA EXAMINADORA \\
			\vspace{6mm}
			\rule{11cm}{.1mm} \\
			Prof. DRA. LAURA COSTA SARKIS - Orientador, UFAC \\
			\vspace{6mm}
			\rule{11cm}{.1mm} \\
			Prof. <NOME DO AVALIADOR>, <INSTITUI\c{C}\~AO>\\
			\vspace{6mm}
			\rule{11cm}{.1mm} \\
			Prof. <NOME DO AVALIADOR>, <INSTITUI\c{C}\~AO>\\
			\vspace{4mm}
			\rule{11cm}{.1mm} \\
			Prof. <NOME DO AVALIADOR>, <INSTITUI\c{C}\~AO>\\
			\vspace{6mm}
			
		\end{center}
	}
\end{flushright}
\begin{center}
	\vspace{4mm}
	RIO BRANCO - ACRE \\
	%\vspace{6mm}
	2020
	
\end{center}
\clearpage
% ---
% Dedicatória
% ---
\begin{dedicatoria}
   \vspace*{\fill}
   \centering
   \noindent
   \textit{ Dedicat\'oria: Este trabalho não seria possível sem a educação que me foi concedida, os ensinamentos e a sabedoria oriundos de você, Lucimar do Rego Albuquerque de Lima, muito mais que uma mãe, uma inspiração. <IN MEMORIAN>.} \vspace*{\fill}
\end{dedicatoria}
% ---

% ---
% Agradecimentos
% ---
\begin{agradecimentos}
\lipsum[3]
\end{agradecimentos}
% ---
\clearpage
% ---
% Epígrafe
% ---
\begin{epigrafe}
    \vspace*{\fill}
	\begin{flushright}
		``Assim como casas são feitas de pedras, a ciência é feita de fatos. Mas uma pilha de pedras não é uma casa e uma coleção de fatos não é, necessariamente, ciência''.  \\ \textbf{Jules Henri Poincaré}, matemático, físico e filósofo da ciência francês
	\end{flushright}
\end{epigrafe}
% ---

% ---
% RESUMOS
% ---

% resumo em português
\setlength{\absparsep}{18pt} % ajusta o espaçamento dos parágrafos do resumo
\begin{resumo}
A quantidade de informações disponíveis cresce a cada ano. Aumenta, junto com o volume de dados e informações, o interesse em tratar, analisar e descobrir conhecimento a partir desta avalanche de dados. A mineração de dados juntamente com o aprendizado de máquina são duas ferramentas-chave dentro deste processo de descoberta de conhecimento e utilização de todos esses dados e informações para propósitos úteis. Entretanto, antes que os dados possam ser analisados, eles precisam ser armazenados e os sistemas computacionais sofrem, também de forma crescente, permanentes ameaças à sua segurança. Neste contexto se inserem as políticas para sistemas computacionais que buscam garantir meios para proteção, confidencialidade e confiabilidade dos acessos dos usuários aos objetos dentro dos sistemas de uma organização. Em sistemas com múltiplos sujeitos, muitas ações e diversos objetos, eventualmente, ocorrerão conflitos entre políticas. Um conflito ocorre quando os objetivos de duas ou mais políticas não podem ser atendidos simultaneamente em determinado contexto. Este trabalho tem como objetivo propor que o problema da detecção de conflitos em políticas pode ser convertido em um problema de \textit{data mining} (mineração de dados) resolvido pela tarefa da classificação além de modelar e sintetizar uma forma de detectar estes conflitos mediante o uso de diferentes algoritmos e técnicas da aprendizagem de máquina que consigam acurácias elevadas e forneçam modelos genéricos o suficiente para serem usados em outros contextos.

 \textbf{Palavras-chave}: Controle de Acesso. Mineração de dados. Aprendizagem de máquina. Conflitos diretos. Conflitos indiretos. Detecção de conflitos.
\end{resumo}

% resumo em inglês
\begin{resumo}[Abstract]
 \begin{otherlanguage*}{english}
   The amount of information available grows every year. It increases, along with the volume of data and information, the interest in treating, analyzing and discovering knowledge from this avalanche of data. Data mining associated with machine learning are two key tools within this process of discovering knowledge and using all that data and information for useful purposes. However, before data can be analyzed, it needs to be stored and computer systems are also increasingly threatened with permanent security threats. In this context, policies for computer systems are inserted as a way to guarantee protection, confidentiality and reliability of users' access to objects within an organization's systems. In systems with multiple subjects, many actions and different objects, eventually, conflicts between policies will occur. A conflict occurs when the objectives of two or more policies cannot be met simultaneously in a given context. This work aims to propose that the problem of detecting conflicts in policies can be converted into a problem of data mining (data mining) solved by the task of classification, in addition to modeling and synthesizing a way of detecting these conflicts through the use of different machine learning algorithms and techniques that achieve high accuracy and provide generic models to be used in other contexts.

   \vspace{\onelineskip}
 
   \noindent 
   \textbf{Keywords}: Access control. Data mining. Machine learning. Direct conflicts. Indirect conflicts. Conflict detection.
 \end{otherlanguage*}
\end{resumo}

% ---
% inserir lista de ilustrações
% ---
\pdfbookmark[0]{\listfigurename}{lof}
\listoffigures*
\cleardoublepage
% ---

% ---
% inserir lista de quadros
% ---
%\pdfbookmark[0]{\listofquadrosname}{loq}
%\listofquadros*
%\cleardoublepage
% ---

% ---
% inserir lista de tabelas
% ---
\pdfbookmark[0]{\listtablename}{lot}
\listoftables*
%\cleardoublepage
% ---

% ---
% inserir lista de abreviaturas e siglas
% ---
\begin{siglas}
  \item[AM] \textit{Aprendizado de Máquina}
  \item[ATM] \textit{Air-Traffic Management}
  \item[BPNN] \textit{Basic Probabilistic Neural Network}
  \item[CPNN] \textit{Constructive Probabilistic Neural Network}
  \item[DAC] \textit{Discretionary Access Control}
  \item[DARPA] \textit{Defense Advanced Research Projects Agency}
  \item[DoS] \textit{Denial of Service}
  \item[GPGU] \textit{General Purpose Graphic Processor Unit}
  \item[GPU] \textit{Graphic Processor Unit}
  \item[KDD] \textit{Knowledege Discovery in Data Bases}
  \item[MAC] \textit{Mandatory Access Control}
  \item[ML] \textit{Machine Learning}
  \item[MLP] \textit{MultiLayer Perceptron}
  \item[NFL] \textit{No-Free Lucnch}
  \item[PMC] \textit{Perceptron MultiCamadas}
  \item[RBAC] \textit{Role-Based Access Control}
  \item[ReBAC] \textit{Relationship-Based Access Control}
  \item[RNA] \textit{Rede Neural Artificial}
  \item[RTLS] \textit{Real Time Location Systems}
  \item[SMA] \textit{Sistema Multi Agentes}
  \item[SVM] \textit{Support Vector Machiner}
\end{siglas}
% ---

% ---
% inserir lista de símbolos
% ---
%\begin{simbolos}
%  \item[$ \Gamma $] Letra grega Gama
%  \item[$ \Lambda $] Lambda
%  \item[$ \zeta $] Letra grega minúscula zeta
%  \item[$ \in $] Pertence
%\end{simbolos}
% ---

% ---
% inserir o sumario
% ---
\pdfbookmark[0]{\contentsname}{toc}
\tableofcontents*
\cleardoublepage
% ---


% ----------------------------------------------------------
% ELEMENTOS TEXTUAIS
% ----------------------------------------------------------
\textual

\chapter{Introdução} \label{introducao}
Neste capítulo introdutório serão descritos o problema da pesquisa, a justificativa, a hipótese do trabalho, os objetivos, as soluções propostas, os resultados esperados, as limitações da pesquisa e como este trabalho está organizado.\index{sinopse de capítulo}
\section{Contextualização} \label{contextualizacao}
De acordo com \citeonline{alecrim2019}, o volume de dados e informações cresce exponencialmente a cada ano, portanto, há uma frequente e ininterrupta demanda por mais infraestrutura de TI nas empresas, nos governos e mesmo nos usuários domésticos e, mais ainda, por um correto tratamento, destino e interpretação à imensidão de dados gerados por pessoas, empresas e governos.\cite{machado2014} 

Em uma ampla variedade de campos, os dados estão sendo coletados e acumulados em um ritmo acelerado e há, assim, uma crescente demanda por análise adequada destes.\cite{fayyad1996, lima_fraud_2012}. Neste contexto se insere a mineração de dados com suas técnicas para tratamento e extração de conhecimento desse volume crescente de dados.\cite{Boscarioli2017, ferrari2017}

Este trabalho usa diversas tarefas da mineração de dados para modelar uma hipótese que possibilite detectar conflitos em políticas de controle de acesso. Para isso, diversos algoritmos de classificação serão explorados, descritos e utilizados com ênfase nas redes neurais e outras técnicas lineares de classificação. 

As políticas de proteção, confidencialidade e confiabilidade da informação, como as de controle de acesso, sendo parte da área de segurança computacional, são uma das formas de garantir, mediante o estabelecimento de regras, padrões e normas a salvaguarda e a disponibilidade das informações dos sistemas.\cite{sarkis2017}\cite{bui_efficient_2019}. 

Este tema, cf. \citeonline[p.1]{ueda_tese_2012} ``é um tema de pesquisa importante dentro do contexto de segurança de sistemas, pois é um dos componentes fundamentais em qualquer sistema de computação''.

Segundo \citeonline{li_security_2006}, um aspecto muito relevante e muitas vezes tratado com pouca ênfase na construção de sistemas é a formulação, gerenciamento e manutenção de políticas de segurança da informação, principalmente as de controle de acesso. 

Nas palavras de \citeonline[p.1]{ueda_tese_2012},
\begin{citacao}
	A definição dessas políticas é normalmente orientada por modelos que fornecem um conjunto de regras e mecanismos para o funcionamento seguro de uma representação abstrata de sistemas. Porém, a administração de tais políticas frequentemente \textit{se torna um processo complexo}, pois deve garantir que elas sejam eficientes e que não comprometam o \textit{desempenho} dos sistemas. \emph{[Grifo do autor.]}
\end{citacao}

Uma política, como as de controle de acesso, descrevem qual ação um sujeito (em um sistema) pode fazer (\textit{permissão}), não pode fazer (\textit{proibição}) ou é obrigado a fazer (\textit{obrigação}) sobre um objeto em um dado contexto \cite{sarkis2017}. 

De maneira semelhante são conceituadas as \textit{normas} e os conjuntos de normas usados para lidar com a autonomia e a diversidade de interesses entre os diferentes agentes em um sistema multiagentes como o descrito e estudado em \citeonline{eduardo2017}. 

Essas normas que regulam as ações dos agentes são análogas às definições das políticas citadas anteriormente nesta \autoref{contextualizacao} e são, cf. \citeonline{eduardo2017} e \citeonline{sarkis:artigo:2016} fatores importantes para garantir a eficácia da segurança dos sistemas.

Em contextos reais, porém, muitas vezes as políticas de segurança (e \textit{normas}) apresentam conflitos entre si. Estes surgem quando, por exemplo, duas políticas regulando o mesmo comportamento de determinado objeto em um sistema estão ativas, mas uma delas obriga (ou permite) a realização de determinado comportamento ou ação enquanto a outra proíbe o mesmo. \cite{sarkis2017}\cite{eduardo2017}. 

A detecção automatizada destes conflitos, com alta acurácia e custo computacional conveniente, é o problema de pesquisa deste trabalho. A descrição pormenorizada dos conflitos entre políticas encontra-se na seção \ref{deteccao_conflitos} deste trabalho.
\subsection{Problema}\label{problema}
O problema investigado neste trabalho consiste na \textit{detecção de conflitos de forma automatizada usando técnicas de mineração de dados e aprendizagem de máquina} que apresentem acurácias superiores a 95\% e que, ao analisar simultaneamente várias políticas ao se inserir uma nova instância não leve a um custo computacional exponencial.
\subsection{Justificativa}\label{justificativa}
Na detecção de conflitos em políticas, geralmente, cf. revisão da literatura ( no \autoref{referencial_teorico}), usam-se abordagens como as de \citeonline{sarkis2017} baseadas em análise de ontologias entre os atributos que compõem uma política e regras de propagação destas políticas ou procedimentos como os descritos em \citeonline{eduardo2017} que utilizam lógica deôntica\footnote{A lógica \textit{deôntica} é um tipo de lógica usada para analisar de modo formal as normas e as proposições que tratam dessas normas \cite{eduardo2017}} para encontrar os conflitos. Estas duas abordagens definem tipos de conflitos que podem ocorrer entre políticas computacionais. Para detectá-los, estas políticas, nestes trabalhos, foram analisadas em pares (e sem filtros para agrupamentos) e mesmo quando foram verificadas múltiplas normas ou políticas \cite{eduardo2017}, toda a base precisa ser ``consultada'' ou ``varrida'', a cada nova instância de uma política inserida ou analisada no sistema (para que o conflito seja ou não detectado).

Para \citeonline{shoham_tennenholtz_1995} esta forma de analisar políticas em pares é um problema NP-completo, ou seja, ainda não foi provado que esta classe de problemas pode ser resolvida em tempo polinomial, sendo assim, são tratados como computacionalmente custosos a cada vez que uma instância nova de política é analisada (em tempo de execução, sendo normalmente exponencial). Consequentemente, com o crescimento orgânico, natural e temporal das políticas em um sistema computacional, a manutenção e gerenciamento das políticas será, eventualmente, computacionalmente oneroso.

Alternativamente, técnicas e algoritmos de aprendizagem de máquina juntamente com as de mineração de dados foram utilizadas com resultados promissores na detecção de conflitos, principalmente em \citeonline{obaidat_multilayer_1994}, \citeonline{chen_flight_2011}, bem como em \citeonline{christodoulou_collision_2008} e \citeonline{jin_development_2002}. Estes estudos abordam problemas variados como detecção de colisões em voos, segurança de acesso computacional, incidentes em rodovias e intrusão de sistemas — todos de alguma forma relacionados à conflitos entre normas, regras, políticas ou direção.

Neste contexto e tendo em vista que: (i) em grandes organizações as políticas de segurança, como as de controle de acesso, pela quantidade de objetos, modalidades, sujeitos e ações inerentes a essas instituições tendem a ter grande quantidade de informações que aumentam diariamente e constantemente nos sistemas computacionais \cite{fugini_information_2004}; (ii) que pode ocorrer com a análise de políticas em pares, conforme descrito no trabalho de \citeonline{shoham_tennenholtz_1995}, um problema NP-completo que onera o custo computacional; (iii) que pode-se otimizar conhecimento adquirido e já existente nas organizações (os \textit{datasets} de políticas). 

Propõe-se neste trabalho, aplicar a mineração de dados com técnicas de aprendizagem de máquina, como possibilidade de solução na detecção de conflitos entre políticas computacionais tanto em tempo de design, quanto em tempo de execução, buscando minimizar o custo computacional, ao se aproveitar da ``história'' temporal das políticas da organização, mediante o conhecimento adquirido, ``treinado'' e otimizado pelos algoritmos de aprendizagem de máquina.

%Alterado pela professora
%Na detecção de conflitos em políticas, geralmente, cf. revisão da literatura ( no capítulo \ref{referencial_teorico}),  usam-se abordagens semelhantes as de \citeonline{sarkis2017} baseadas em análise de ontologias entre os atributos que compõem uma política e em regras de propagação das mesmas ou procedimentos como os descritos em \citeonline{eduardo2017} que utilizam lógica deôntica para encontrar os conflitos. 
%Entretanto, para que os conflitos sejam detectados nessas abordagens as políticas são analisadas em pares (e sem filtros para agrupamentos) e mesmo quando são verificadas múltiplas \textit{normas} ou políticas, como em \citeonline{eduardo2017}, todas elas devem ser novamente ``onsultadas'' ou ``varridas'' a cada nova instância de uma política inserida no sistema (para que o conflito seja ou não detectado). 
%O trabalho de \citeonline{shoham_tennenholtz_1995} provou que esta forma de analisar políticas é um \textbf{problema NP-completo}, ou seja, ainda não foi provado que esta classe de problemas pode ser resolvida em tempo \textit{polinomial}, sendo assim, são computacionalmente onerosos a cada vez que uma instância nova de política é analisada (em tempo d execução sendo, normalmente exponencial). Com o crescimento orgânico, natural e temporal da quantidade de políticas em um sistema computacional este fato tende a tornar a manutenção e gerenciamento das políticas algo \textit{computacionalmente custoso e dispendioso}.   
%Conforme se visualiza na seção \ref{trabalhos_relacionados}, as técnicas e algoritmos de aprendizagem de máquina juntamente com as de mineração de dados foram utilizadas com resultados promissores na detecção de conflitos, principalmente em \citeonline{obaidat_multilayer_1994, chen_flight_2011} bem como em \citeonline{christodoulou_collision_2008} e \citeonline{jin_development_2002} que abordam problemas variados como detecção de colisões em voos, segurança de acesso computacional, incidentes em rodovias e intrusão de sistemas --- todos de alguma forma relacionados à conflitos entre normas, regras, políticas ou direção.  
%Em grandes organizações as \textit{políticas de segurança}, como as de controle de acesso, pela quantidade de objetos, modalidades, sujeitos e ações inerentes a essas instituições tendem a ter grande quantidade de informações e elas crescem conforme o uso diário e constante dos sistemas computacionais. Organizações grandes, com múltiplos objetos e ações possuem centenas, às vezes, milhares de políticas (entre elas \textit{políticas de controle de acesso}, por exemplo).\cite{fugini_information_2004}.
%Neste ambiente e considerando que as políticas analisadas em pares, como em \citeonline{sarkis2017}, tem alto custo computacional por este, como visto em \citeonline{shoham_tennenholtz_1995}, \underline{ser um problema NP-completo} e com a estratégia de minimizar a complexidade deste problema otimizando-o baseando suas soluções no conhecimento adquirido e já existente nas organizações (os \textit{datasets} de políticas), a mineração de dados com técnicas de aprendizagem de máquina surgem como possibilidades de solução na detecção de conflitos tanto em tempo de design quanto em tempo de execução, pois, minimiza este custo computacional ao se aproveitar da ``história'' temporal das políticas da organização, mediante o conhecimento adquirido e ``treinado'' e otimizado pelos algoritmos de aprendizagem de máquina.

\subsection{Hipótese}\label{hipótese}
Diante do contexto apresentado na seção anterior e também por \citeonline{fugini_information_2004} tem-se \textit{como hipótese deste trabalho}, que o problema de detectar conflitos entre políticas\textit{ pode ser convertido e transformado} em uma tarefa de classificação da mineração de dados e que o uso de algoritmos de aprendizagem de máquina associados a técnicas de \textit{data mining} para detectar estes conflitos configure um método que apresente precisão e acurácia superiores a 95\%.

\section{Objetivos}\label{objetivos}
Nesta seção serão descritos o objetivo geral e os específicos deste trabalho.
\subsection{Objetivo geral}\label{objetivo_geral}
O objetivo deste trabalho é propor que o problema da detecção de conflitos em políticas pode ser convertido em um problema de \textit{data mining} (mineração de dados) resolvido pela tarefa da classificação além de modelar e sumariar uma forma de detectar estes conflitos mediante o uso de diferentes algoritmos e técnicas da aprendizagem de máquina que consigam acurácias elevadas.
\subsection{Objetivos específicos}
Os objetivos específicos deste trabalho são:
\begin{itemize}	
		\item Estabelecer a relação entre machine learning, técnicas de mineração de dados e o problema do conflito entre políticas;
		\item Determinar e comparar quais algoritmos e técnicas são mais adequados para cada tipo de conflito nas políticas (ou normas) usando as suas acurácias;
		\item Usar e comparar o desempenho, a precisão e taxa de acertos das principais técnicas usadas no aprendizado de máquina: redes neurais artificiais (RNA), Support Vector Machines (SVM) e redes neurais recorrentes e profundas.
		\item Transformar o problema da detecção de conflitos em uma tarefa de classificação da mineração de dados mantendo acurácias elevadas;
		\item Usar \textit{frameworks} de aprendizado de máquina como TensorFlow \cite{kadimisetty_tensorflow_2018} ou Torch \cite{NEURIPS2019_9015}, na construção, treinamento e teste de redes neurais, comparando-os, quando adequado, para o problema específico deste trabalho;
		\item Estabelecer a detecção de conflitos em políticas (ou normas) como uma  classe de problemas a serem resolvidos de forma eficiente por técnicas de aprendizagem de máquina.
\end{itemize}


\section{Solução Propostas}\label{solucao_proposta}
Diante da hipótese apresentada na seção \ref{hipótese}, a solução para o problema apresentado neste trabalho na seção \ref{problema} concentra-se, prioritariamente, em mostrar que \textit{converter} (ou \textit{transformar}) a detecção de conflitos a um \textit{problema de classificação} da mineração de dados associado a técnicas de aprendizagem de máquina, reestruturando os atributos do \textit{dataset}, se necessário, se configura um método com acurácia superior a 95\%.

\textit{A primeira solução} proposta para conflitos diretos entre políticas é usar as técnicas e algoritmos de classificação (aprendizado supervisionado) para  realizar a detecção automatizada de conflitos. Para isso, propõem-se:
\begin{itemize}
	\item Usar, inicialmente, uma rede neural (um perceptron de uma camada ou com somente uma camada oculta e apenas com \textit{forward}) como técnica algorítmica para a detecção de conflitos e
	\item Construir a arquitetura de uma rede neural multicamadas (com camadas ocultas), e retropropagação (\textit{backpropagation}), comparando-a com outro classificador linear, como, por exemplo, o SVM, para estabelecer qual técnica de mineração de dados na detecção de conflitos em políticas é mais precisa.	
\end{itemize}
Realizado os múltiplos experimentos, atestar a hipótese mediante os resultados apresentados.

\section{Método de Pesquisa} 
O método de pesquisa relatando todos os passos necessários para demonstrar os objetivos descritos na seção \ref{objetivos} e de que forma eles foram atingidos serão pormenorizadamente detalhados no capítulo \ref{resultados} deste trabalho.

\section{Resultados Esperados}\label{resultados_esperados}
Ao fim deste trabalho os seguintes resultados são esperados:
\begin{itemize}
	\item Mostrar que o problema da detecção de conflitos em políticas pode ser convertido em um problema de \textit{data mining} (mineração de dados) resolvido pela tarefa da classificação;
	\item Demonstrar que o problema da detecção de conflitos é um prolema linearmente separável;
	\item Mostrar que a política nova (\textit{instância inédita})  é ou não conflitante imediatamente após a criação da mesma usando como base o treinamento da rede neural no \textit{dataset} de políticas existente;
	\item Modelar e resumir uma forma de detectar estes conflitos mediante o uso de diferentes algoritmos e técnicas da aprendizagem de máquina que consigam acurácias superiores a 95\%;
	\item Estabelecer a relação entre machine learning, técnicas de mineração de dados e o problema do conflito entre políticas;
	\item Determinar e comparar quais algoritmos e técnicas são mais adequados para cada tipo de conflito nas políticas (ou normas) usando as suas acurácias;
	\item Usar e comparar o desempenho, a precisão e taxa de acertos das principais técnicas usadas no aprendizado de máquina: redes neurais artificiais (RNA), Support Vector Machines (SVM); redes neurais recorrentes e profundas;
	\item Estabelecer a detecção de conflitos em políticas (ou normas) como uma  classe de problemas a serem resolvidos de forma eficiente por técnicas de aprendizagem de máquina.
\end{itemize} 

\section{Limitações do Trabalho}\label{limitacoes}
Não faz parte do escopo deste trabalho:
\begin{itemize}
	\item Delinear um modelo de política com objetivos semânticos diferenciados. Para os experimentos deste trabalho será usado o modelo de políticas descrito em \citeonline{sarkis2017} e em \citeonline{sarkis:artigo:2016}
	\item Analisar comparativamente os modelos de extensão de políticas em um determinado contexto;
	\item Usar redes neurais convolucionais profundas na detecção dos conflitos;
	\item Abordar a semântica em políticas;	
\end{itemize} 

\section{Organização do trabalho}
Este trabalho está organizado da seguinte forma:
\begin{itemize}	
	\item No \autoref{referencial_teorico} apresenta-se todo o referencial teórico compreendendo uma revisão bibliográfica sobre os principais temas desta proposta de dissertação, como políticas, detecção de conflitos, mineração de dados e aprendizagem de máquina (e seus algoritmos principais)
	\item No \autoref{resultados} são mostrados o método e os experimentos e resultados obtidos.
	\item No \autoref{propostas} mostra-se o cronograma para a finalização da dissertação
	\item No \autoref{conclusoes} apresenta-se as conclusões atingidas e esperadas desta pesquisa.
\end{itemize}
\clearpage
% ----------------------------------------------------------
% PARTE
% ----------------------------------------------------------
%\part{Preparação da pesquisa}
% ----------------------------------------------------------

% ---
% Capitulo com exemplos de comandos inseridos de arquivo externo 
% ---
%\include{abntex2-modelo-include-comandos}
% ---
%\chapter{Introdução} \label{introducao}
\section{Contextualização} \label{contextualizacao}
É fato que o volume de dados e informações cresce exponencialmente a cada ano \cite{alecrim2019}, portanto, há uma frequente e ininterrupta demanda por mais infraestrutura de TI nas empresas, nos governos e mesmo nos usuários domésticos \cite{machado2014} e, mais ainda, por um correto tratamento, destino e interpretação à imensidão de dados gerados por pessoas, empresas e governos. 

Em uma ampla variedade de campos, os dados estão sendo coletados e acumulados em um ritmo dramático.\cite{fayyad1996}\cite{lima_fraud_2012}. Há portanto esta crescente demanda por tratamento adequado a estes dados. Neste contexto se insere a mineração de dados com suas técnicas para tratamento e extração de conhecimento desse grande volume de dados.\cite{Boscarioli2017}\cite{ferrari2017}

Este trabalho usa diversas tarefas da mineração de dados para modelar uma hipótese que possibilite detectar conflitos em políticas de controle de acesso. Para isso, algoritmos de classificação serão explorados, descritos e utilizados.

As políticas de controle de acesso, como parte da segurança computacional, são uma das formas de garantir a proteção das informações dos sistemas.\cite{sarkis2017}

O controle de acesso, cf. \cite[p.1]{ueda_tese_2012} ``é um tema de pesquisa importante dentro do contexto de segurança de sistemas, pois é um dos componentes fundamentais em qualquer sistema de computação."

Segundo \cite{li_security_2006}, um aspecto muito relevante e muitas vezes tratado com pouca ênfase na construção de sistemas é a formulação de políticas de controle de acesso. 

Nas palavras de \cite[p.1]{ueda_tese_2012},
\begin{quotation}
	A definição dessas políticas é normalmente orientada por modelos que fornecem um conjunto de regras e mecanismos para o funcionamento seguro de uma representação abstrata de sistemas. Porém, a administração de tais políticas frequentemente \textit{se torna um processo complexo}, pois deve garantir que elas sejam eficientes e que não comprometam o desempenho dos sistemas. \textbf{\emph{[Grifo do autor.]}}
\end{quotation}

Uma política de controle de acesso descreve qual  ação um sujeito pode fazer (permissão), não pode fazer (proibição) ou é obrigado a fazer (obrigação) sobre um objeto em um dado contexto\cite{sarkis2017}

Muitas vezes as políticas de segurança apresentam conflitos. A detecção destes conflitos é o problema de pesquisa deste trabalho.


\section{Controle de Acesso} \label{controle_acesso}
A política de segurança em um sistema computacional garante a proteção de suas informações. Dentre as tecnologias utilizadas para assegurar essas propriedades, temos o controle de acesso. \cite{sarkis:artigo:2016}

O controle de acesso é o mecanismo central para atingir os requisitos de segurança em  sistemas  de  informação. \cite{wang_conflicts_2010}. Dessa forma,  trata-se  de  uma  tecnologia indispensável para quem faz uso de qualquer tipo de sistema, podendo basear-se ou coexistir com outros serviços de segurança. \cite{sandhu:1996}.

Os modelos de controle de acesso fornecem um conjunto de regras e mecanismos para o funcionamento seguro dos sistemas, sendo responsáveis pela definição de políticas de controle de acesso. As políticas são diretrizes de alto nível que determinam como os acessos são controlados e decisões de acessos são estabelecidas. \cite{di_vimercati_policies_2005} \cite{sarkis2017}	


\subsection{Políticas de Controle de acesso}
As políticas de controle de acesso tradicionais, inicialmente foram chamadas de autorizações e tinham a seguinte forma: {sujeito, objeto, operação}. Estas autorizações especificavam as operações que os sujeitos podiam executar sobre os objetos. \cite{di_vimercati_policies_2005}\cite{sarkis2017}

Uma política de controle de acesso tem como objetivo definir ou limitar o comportamento atual ou futuro de objetos para garantir que as suas ações estejam alinhadas com os objetivos da empresa.\cite{dunlop_dynamic_2002}\cite{sarkis2017}


\section{Detecção de conflitos} \label{deteccao_conflitos}

Os conflitos podem ocorrer quando diferentes conjuntos de condições resultam em permitir e negar simultaneamente, ao mesmo papel, à mesma solicitação, ou proibir e obrigar o mesmo papel, à mesma solicitação.

Diz-se que duas regras estão em conflito quando o cumprimento de uma das regras viola a outra e vice-versa. 

Ex:

{\scriptsize \texttt{ \{P1= {\textbf{Permitido}, na Universidade, Paulo, acessar processos administrativos\} }}}

{\scriptsize \texttt{ \{P2= {\textbf{Proibido}, na Universidade, Paulo, acessar processos administrativos\} }}}

A capacidade de um sistema reconhecer um estado inconsistente em andamento ou em potencial é denominada \textbf{detecção de conflitos.}

Em um conflito indireto, as políticas conflitantes regulam ações diferentes (mas relacionadas) executadas por diferentes sujeitos (porém, relacionados) sobre objetos diferentes (mas, relacionados) em organizações diferentes (mas, relacionadas). 

Além disso, um conflito indireto pode ainda ocorrer, mesmo quando as políticas em conflito não têm modalidades contraditórias ou contrárias.

Ex:

{\scriptsize \texttt{P3 = {Obrigado, Empresa E, \textbf{Funcionário, receber}, avaliação, mensal}}}

{\scriptsize \texttt{P4= {Permitido, Empresa E, \textbf{Analista, conceder}, avaliação, mensal}}}

Este conflito não seria detectado diretamente, porém há um conflito se considerarmos os relacionamentos.

\section{Problema}\label{problema}
O problema investigado neste trabalho consiste em \textit{detectar conflitos diretos e indiretos usando técnicas de mineração de dados e aprendizagem de máquina}. É fato que, por exemplo, em \textit{logs} de controle de acesso, a quantidade de informações é grande e cresce exponencialmente conforme o uso. Sistemas grandes com múltiplos objetos e ações tem centenas, às vezes, milhares de políticas.\cite{fugini_information_2004}

Tem-se \textit{como hipótese deste trabalho}, portanto, diante do contexto apresentado por \cite{fugini_information_2004} que, o uso de técnicas de \textit{data mining}(mineração de dados) associada a técnicas de aprendizagem de máquina para detectar os conflitos sejam eles diretos ou indiretos entre diversas políticas de controle de acesso seja uma técnica adequada com acurácias superiores a 95\%.

\subsection{Objetivo geral}\label{objetivo_geral}
O objetivo deste trabalho é modelar uma forma de detectar, mediante o uso de técnicas de mineração de dados (e aprendizagem de máquina) os conflitos, sejam eles, diretos ou indiretos entre as políticas de controle de acesso de um sistema. 

O objetivo \textit{específico} deste estudo orientado é realizar uma pesquisa inicial que desse suporte teórico e prático à dissertação. Neste primeiro momento buscou-se estabelecer a relação entre machine learning, técnicas de mineração e conflito entre políticas.

\subsection{Solução Proposta/Hipótese}\label{solucao_proposta}
A solução para o problema apresentado neste trabalho é \textit{converter} (ou \textit{reduzir}) a detecção de conflitos a um \textit{problema de classificação} da mineração de dados/aprendizagem de máquina reestruturando os atributos (colunas) se necessário. 

\textit{A primeira solução/hipótese} (para conflitos diretos) é usar as técnicas e algoritmos de classificação (aprendizado supervisionado) para tentar realizar a classificação automática para resolução de conflitos (comparando-os entre si). 

\textit{A segunda solução/hipótese} é usar uma rede neural (um perceptron de uma camada) como técnica algorítmica para a detecção de conflitos diretos. 

\textit{A terceira solução/hipótese} é usar uma rede neural muticamadas (ou outro classificador como SVM) para a detecção de conflitos indiretos em políticas de controle de acesso.
 
\chapter{Referencial teórico}\label{referencial_teorico}
Nesta seção, alguns conceitos da pesquisa bibliográfica realizada serão explanados com o intuito de atingir os objetivos e o delimitar o escopo deste trabalho. Serão abordados modelos de políticas de controle de acesso e conflitos entre as mesmas e alguns trabalhos relacionados sobre o tema. Além de uma revisão bibliográfica acerca de mineração de dados, aprendizagem de máquina, algoritmos de classificação(com destaque para as Redes Neurais e \textbf{SVM} --- \textit{Support Vector Machines} que fazem parte da hipótese deste trabalho).
\subsection{Modelos de políticas de controle de acesso}\label{modelo_politicas}
\cite{moffett_policy_1994} consideram que um modelo de política deve ter pelo menos os seguintes atributos: \textit{modalidade, sujeito, objeto e ação}. A modalidade da política envolve definir uma autorização, uma permissão ou uma proibição. O sujeito da política é a quem ela é direcionada. O objeto define o conjunto de objetos, no qual a política está direcionada. A ação é especificada como operações que podem ser executadas em objetos no sistema \cite{moffett_policy_1994}. Os atributos da política apresentada por estes autores são considerados em outros trabalhos, com a mesma conotação aqui utilizada, para a definição de uma política.\cite{sarkis2017}

\subsection{Trabalhos Relacionados}\label{trabalhos_relacionados}
\cite{bui_efficient_2019} discorre acerca de mineração de dados em políticas de controle de acesso, especificamente do modelo RBAC (\textit{role-based access control}). \cite{lupu_conflicts_1999} discorre sobre conflitos no gerenciamento de sistemas distribuídos com base em políticas de controle de acesso. \cite{koch_conflict_2002}, estudou a detecção e resolução de conflitos nas especificações das políticas de controle de acesso. \cite{neri_conflict_2012}, abordou a detecção de conflitos em políticas de segurança usando a tecnologia da Web Semântica.

\cite{obaidat_multilayer_1994} aborda um sistema de rede neural multicamada para segurança de acesso a computadores. \cite{christodoulou_collision_2008} aborda a detecção de conflitos sobre a ótica da prevenção de colisões no voo livre de aeronaves comerciais usando redes neurais e programação não linear. \cite{mukkamala_intrusion_2002} estuda a detecção de invasões usando redes neurais e máquinas de vetores de suporte. Neste último trabalho a ideia é descobrir padrões úteis ou características que descrevam o comportamento intrusivo de um usuário em um sistema e os autores usam este conjunto de características para construir classificadores que puderam reconhecer anomalias e intrusões conhecidas. 

\cite{jin_development_2002} aborda o desenvolvimento e adaptação de redes neurais probabilísticas construtivas na detecção de incidentes em rodovias. \cite{debar_neural_1992} estuda um componente de rede neural para um sistema de detecção de intrusão. Ainda em detecção de conflitos \cite{chen_flight_2011} trabalha com a detecção e resolução de conflitos de vôo com base em redes neurais.

Todos estes trabalhos foram base para o estudo, amadurecimento bibliográfico e aprofundamento teórico sobre o problema e as soluções propostas neste trabalho, principalmente aqueles relacionados a intrusões e detecção de conflitos aéreos pois não só serviram de inspiração como provavelmente as técnicas possam ser extrapoladas para o uso na detecção de conflitos, tema deste estudo.

Além de \cite{sarkis2017} e \cite{sarkis:artigo:2016} que são os trabalhos- base do estudo da detecção de conflitos, da determinação do modelo das políticas utilizadas e analisadas e das definições utilizadas neste trabalho. Todas as hipóteses e problemas estudados por este trabalho derivam do estudo inicial de \cite{sarkis2017}. 

\subsection{Modelo de Política utilizado}\label{modelo_politica_utilizada}
Cf. \cite{sarkis2017}:

\begin{quotation}
	Definir uma política de controle de acesso não é uma tarefa simples, principalmente porque algumas vezes é necessário representar formalmente políticas complexas, tais como as que tem origem em práticas de leis e regulamentos organizacionais.
\end{quotation} 

Desta forma, a definição da política deve combinar todos os diferentes regulamentos para ser executada e considerar todas as possíveis ameaças adicionais relativas ao uso de sistemas. \cite{di_vimercati_policies_2005}

O modelo de política utilizado neste trabalho baseia-se inteiramente no modelo proposto por \cite{sarkis2017} e \cite{sarkis:artigo:2016}.

Portanto, de acordo com \cite[p.36]{sarkis2017}:

\begin{quotation}
	Uma política é uma tupla da forma:
	\begin{equation}\label{key}
	Policy = KP \times Org \times SR \times AA \times OV \times Ac \times Dc
	\end{equation}
	Onde $ KP $ descreve o tipo de política (uma proibição (F), da palavra em inglês Forbidden; uma permissão (P); ou uma obrigação (O)). $ Org $. relata o local (ambiente) onde a política deve ser cumprida, isto é, a organização na qual os sujeitos devem cumprir a política. $ SR $ descreve a quem (entidades) se destina a política (pode ser um sujeito $ s \in S $ ou um papel $ r \in R $, ou seja, $ SR = S \cup R $. Um sujeito pode ser um usuário $ u \in U $ ou uma organização $ org \in Org $ representando o grupo de sujeitos que devem cumprir com a política, isto é, $ S = U \cup Org $). $ AA $ identifica uma ação $ a \in A  $ ou uma atividade $ act \in Act  $ (uma atividade é a união de várias ações relacionadas). $ OV $ relata um objeto $ o \in O $ ou uma visão $ v \in V $ que está sendo manipulada pela ação/atividade (uma visão é a união de vários objetos). $ Ac $ é a condição de ativação da política e $ Dc $ é a condição de desativação da política. Uma condição constitui a configuração para um evento, em termos de que a política deva seguir. 
\end{quotation}

Em \cite{sarkis2017} definiu-se $ Ac $ e $ Dc $ como datas, desta forma $ Ac $ é a data de ativação da política e $ Dc $ é a data de desativação.

A figura \ref{fig:modelo_politica} exemplifica o modelo de políticas utilizadas neste estudo para a mineração de dados e aprendizagem de máquina.

\begin{figure}[h!]
	\centering
	\includegraphics[width=.8\textwidth]{imagens/modelo_politica.png}
	\caption{Modelo das políticas utilizadas no estudo}
	{\scriptsize Fonte: compilação do autor}
	\label{fig:modelo_politica}
\end{figure}

\subsection{Aprendizagem de máquina} \label{aprendizagem_maquina}
Aprendizado de máquina ou \textit{machine learning} é um braço da Inteligência Artificial que emprega técnicas e algoritmos na criação de modelos computacionais dos quais a característica princial é a capacidade de descobrir padrões em um grande volume de dados ou de melhorar o desempenho de uma determinada tarefa através da experiência (do \textit{reforço}).\cite{mohri_foundations_2018} \cite{alpaydin_introduction_2014} \cite{swamynathan_mastering_2019}

Nas palavras de Arthur Lee Samuel, considerado um dos pioneiros na área de inteligência artificial \cite{wiederhold_arthur_1992}, aprendizado de máquina é ``o campo de estudo que dá aos computadores a capacidade de aprender sem serem explicitamente programados". \cite[p. 89]{simon_too_2013}.

Aprendizado de máquina tem sido aplicado na automatização de funções que para os humanos são executadas intuitivamente, mas que são difíceis de definir formalmente. \cite{sarkar_2017}

Assim, de forma geral, a aprendizagem de máquina tem por objetivo estudar e desenvolver métodos computacionais para obter sistemas capazes de adquirir conhecimento de forma automatizada. \cite{lima_ia_2016}

A capacidade de determinados algoritmos tem de aprender a partir de exemplos é chamado de \textbf{aprendizado indutivo}. Estes algoritmos aprendem relacionamentos eventualmente existentes entre os dados, mostrando o resultado nos modelos de conhecimento gerado. \cite{goldschmidt2005}\cite{alpaydin_introduction_2014}

As principais abordagens de aprendizado que determinam os 3 principais tipos de aprendizagem são: a aprendizagem supervisionada, a aprendizagem não-supervisionada e a aprendizagem por reforço. \cite{Norvig2013}. 

Na \textbf{aprendizagem não-supervisionada} o modelo/hipótese busca padrões na entrada, embora não seja fornecido nenhum feedback explícito. Portanto, na abordagem não-supervisionada não há, nos dados, uma classe, não há um rótulo prévio, ou seja, não existe a informação da saída desejada. O processo de aprendizado busca identificar regularidades entre os dados e não é necessária a divisão prévia dos dados em dados de treinamento, validação e teste.  A tarefa mais comum de aprendizagem não supervisionada é o agrupamento. \cite{Norvig2013} \cite{Boscarioli2017} \cite{goldschmidt2005} \cite{aprenda_mineracao_fernando_amaral16}

Na \textbf{aprendizagem supervisionada} o modelo/hipótese observa alguns exemplos de pares de entrada e saída, e aprende uma função que faz o mapeamento entre a entrada e a saída. Portanto, ela compreende a abstração de um modelo a partir dos dados apresentados na forma de pares ordenados (\textit{entrada, saída, saída desejada}). Há, assim, uma \textit{classe}, ou um atributo especial com o qual se pode comparar e validar o resultado.

Na \textbf{aprendizagem por reforço}, aprende-se a partir de uma série de reforços --- recompensas ou punições. Não está disponível, geralmente, na aprendizagem por reforço, para o algoritmo de aprendizado de máquina, um conjunto de dados para treinamento. O aprendizado se dá, então, pela interação com o ambiente que se deseja atuar por um determinado período com o objetivo de melhorar o desempenho de uma determinada tarefa. \cite{Norvig2013} \cite{aprenda_mineracao_fernando_amaral16} \cite{silva_restaurante_2019}	


\subsection{Mineração de Dados}\label{mineracao_dados}
Uma das características de nossa era é produção de dados em grande volume, velocidade e variedade de todas as formas, por dispositivos espalhados em toda parte. Entretanto, dados, mesmo em grande quantidade, são apenas dados. É preciso produzir informação e conhecimento para explorar as vantagens que essa massa pode trazer. O dado necessita ser, de alguma forma, analisado, tratado para que informações e conhecimento possa ser extraído. \cite{aprenda_mineracao_fernando_amaral16} \cite{ferrari2017}

Conforme \cite{fayyad1996}:
\begin{quotation}
	[...] Os computadores permitiram que os humanos coletassem mais dados do que podemos digerir, é natural [,portanto] recorrer a técnicas computacionais para nos ajudar a desenterrar padrões e estruturas significativas a partir dos volumosos volumes de dados. Por isso, [a mineração de dados] é uma tentativa de resolver um problema que a era da informação digital transformou em realidade para todos nós: sobrecarga de dados.
\end{quotation}

Para \cite{Boscarioli2017},
\begin{quotation}
	De forma simplificada, a mineração de dados pode ser definida como um processo automático ou semiautomático de explorar analiticamente grandes bases de dados, com a finalidade de descobrir padrões relevantes que ocorrem nos dados e que sejam importantes para embasar a assimilação de informação importante, suportando a geração de conhecimento. 
\end{quotation}

Ainda, segundo \cite{fayyad1996}, 
\begin{quotation}
	O termo mineração de dados tem sido usado principalmente por estatísticos, analistas de dados e comunidades de sistemas de informações gerenciais (MIS). Ele também ganhou popularidade no campo do banco de dados. O termo descoberta de conhecimento em bancos de dados [(KDD, da sigla em Inglês)] foi cunhada no primeiro workshop do KDD em 1989 para enfatizar que o conhecimento é o produto final de uma descoberta baseada em dados. Foi popularizado nos campos de IA e aprendizado de máquina.
\end{quotation}

Dessa forma, a mineração de dados é parte integrante de um processo mais amplo, conhecido como descoberta de conhecimento em bases de dados (\textit{Knowledge Discovery in Databases}, ou \textit{KDD})\cite{fayyad1996}. Embora se use \textit{mineração de dados} como sinônimo de KDD, a terminologia é empregada para a etapa de \textit{descoberta}  do processo de KDD, que inclui a \textit{seleção} e \textit{integração} das bases de dados, a \textit{limpeza} da base, a \textit{seleção e transformação} dos dados, a \textit{mineração}(propriamente) e a \textit{avaliação} dos dados. \cite{ferrari2017}\cite{Boscarioli2017}.

Assim, a mineração de dados  é definida em termos de esforços para a descoberta de padrões em bases de dados. A partir destes padrões descobertos, há condições de se gerar conhecimento útil para um processo de tomada de decisão (ou a geração de conhecimento para esta tomada).

O KDD (Knowledge Discovery in Database) é um processo de busca de conhecimento em bancos de dados e, de modo geral, consiste de uma sequência iterativa de passos(ou \textbf{etapas})\footnote{O processo de KDD, segundo \cite{fayyad1996} é composto por: \textit{Seleção de dados; Pré-processamento; Transformação; Mineração; Análise e assimilação de resultados}}: limpeza de dados, integração dos dados, Seleção, Transformação e Mineração dos dados, Avaliação dos padrões e Apresentação e Assimilação do conhecimento. Este processo é iterativo e, em alguma etapa, pode-se voltar para uma anterior. \cite{Boscarioli2017}

Neste trabalho as tarefas de seleção e transformação dos dados farão parte da etapa chamada de pré-processamento (cf. \cite{Boscarioli2017} e serão descritas na seção \ref{resultados}.

O termo \textbf{modelo de conhecimento}(ou hipótese) é utilizado na literatura (e neste trabalho) para fazer referência a um padrão ou conjunto de padrões descobertos (que é, enfim, o \textit{propósito} do processo de KDD). Estes padrões são conhecimentos representados segundo as normas sintáticas de alguma linguagem formal. Estes padrões podem ser classificados em dois tipos: preditivos e descritivos. O intuito dos preditivos é resolver um problema específico de prever os resultados ou valores de um ou mais atributos, em função dos valores de outros atributos. Os descritivos (ou informativos) tem o intuito de apresentar informações interessantes e importantes sobre os dados que um especialista de domínio possa não conhecer. Modelos de conhecimento compostos exclusivamente por padrões preditivos são chamados de modelos preditivos, enquanto que modelos descritivos são modelos de conhecimento compostos por padrões descritivos. \cite{goldschmidt2005}\cite{ferrari2017}\cite{Boscarioli2017}

Neste contexto, este trabalho tem como objetivo modelar uma forma de detectar, mediante o uso de técnicas da mineração de dados (e aprendizagem de máquina) os conflitos entre as políticas de controle de acesso de um sistema. Vários algoritmos e técnicas serão utilizados sendo que eles serão devidamente analisados usando-se métricas específicas para cada algoritmo.


\subsection{Algoritmos de classificação}
Conforme \cite{Rocha2012}, ``o termo \textit{classe} deve ser usado quando existe informação sobre quantas e quais são as partições presentes em um conjunto de dados, bem como qual exemplar pertence a qual partição".

Comumente denomina-se \textit{classificação} o processo pelo qual se determina uma função de mapeamento capaz de indicar a qual classe pertence algum exemplar de um domínio sob análise, baseando-se em um conjunto já classificado. \cite{Boscarioli2017}.

Assim, classificação é uma técnica de mineração de dados (aprendizado de máquina) usada para prever a associação ao grupo para instâncias de dados.\cite{classification2013}. É, segundo \cite{aprenda_mineracao_fernando_amaral16} e \cite{performance_classification2013}, a tarefa mais utilizada em mineração de dados. Além de ser a mais complexa e a que possui a maior quantidade de algoritmos disponíveis.\cite{classification2013}	

A classificação é uma das tarefas preditivas de Mineração de Dados e aprendizado de máquina. Tarefas de predição consistem na análise de um dataset (conjunto de dados), descritos por atributos e rótulos associados com o objetivo de descobrir um modelo capaz de mapear corretamente cada um dos dados a seus rótulos. Esse objetivo é alcançado por meio de técnicas chamadas de supervisionadas. A análise preditiva é dividida em categórica, também chamada de classificação ou em numérica, também chamada de regressão. \cite{Boscarioli2017} \cite{classification2013} \cite{ferrari2017} \cite{goldschmidt2005}

Formalmente, a tarefa de classificação pode ser descrita como a busca por uma função de mapeamento para um conjunto $X$ de vetores de entrada (ou, exemplares --- os dados) $\vec{x_i} \in E^d$ para um conjunto finito de rótulos $C$ de cardinalidade $c$. A função $F$ é, então, definida como $F: E^d \times W \rightarrow C$, em que $d$ é a dimensão do espaço $E$, ou seja, a quantidade de coordenadas do vetor $\vec{x_i}$, e $W$ é um espaço de parâmetros ajustáveis por meio do algoritmo de indução supervisionada. \cite{Boscarioli2017}

Pode ser dividida em, ao menos, duas categorias: classificação binária e classificação multiclasse. Na binária, a cardinalidade $c$ é 2. Para o caso em que $c > 2$, o problema é considerado de múltiplas classes.\cite{Boscarioli2017}

Os textos de \cite{classification2013}, \cite{performance_classification2013}, \cite{Wolpert:1996}, \cite{classification_survey2012} e \cite{using_data_mining2012}  trazem reflexões, técnicas, comparações e explicações detalhadas de muitos algoritmos de classificação, entre eles, árvores de decisão, k-vizinhos mais próximos, Naive Bayes e Redes Bayesianas, Redes Neurais Artificiais, Máquinas de Vetores de Suporte (SVM) entre outros.   

Sobre teoria da aprendizagem e algoritmos de classificação há em \cite{Norvig2013} uma discussão sobre qual seria, em relação às hipóteses de modelos de aprendizagem, aquela (ou aquelas) que melhor se ajuste aos dados futuros. O autor cita a \textbf{suposição de estacionaridade}, ou seja, que há uma distribuição de probabilidade sobre os dados que permanece estacionária ao longo do tempo. Supõe-se, portanto que cada exemplo de ponto de dados (antes de conhecê-lo) é uma variável aleatória $E_j$ cujo valor observado $e_j = (x_j, y_j)$ é amostrado da distribuição e é independente dos exemplos anteriores. Assim:
\begin{equation}
P(E_j|E_{j-1},E_{j-2}, ... ) = P(E_j), 
\end{equation}
e cada exemplo tem uma distribuição de probabilidade anterior idêntica:
\begin{equation}
P(E_j) = P(E_{j-1}) = P(E_{j-2}) = ... 
\end{equation}
Estes exemplos são chamados de \textit{independentes e identicamente distribuídos} ou \textbf{i.i.d}. Esta suposição é necessária para tentar a previsão sobre o futuro dos dados. Há claro, ainda em \cite{Norvig2013}, um alerta sobre o fato de ser possível a aprendizagem ocorrer caso haja pequenas alterações (lentas) na distribuição.

Outro fato importante para a definição e avaliação da escolha da melhor hipótese (modelo) de um algoritmo de classificação é definir o ``melhor ajuste". \cite{Norvig2013} define a \textbf{taxa de erro} de uma hipótese como uma métrica importante para definir o ``melhor ajuste" de um modelo/hipótese.

A taxa de erro é, assim, a proporção de erros que o algoritmo classificador comete---a proporção de vezes que $h(x)\neq y$ para o exemplo $(x,y)$ --- sendo $h(x)$ a função que mapeia uma hipótese/modelo $h$ com a previsão/valor conhecido $y$. Nem sempre, como alerta, \cite{Norvig2013}, uma hipótese/modelo(algoritmo) $h$ que tenha uma taxa de erro baixa no conjunto de treinamento generaliza bem. A forma de testar o algoritmo é importante. Para isso há, na literatura, algumas técnicas que são utilizadas como estratégia de treinamento, validação e teste.

Autores como \cite{Boscarioli2017}, \cite{aprenda_mineracao_fernando_amaral16}, \cite{data_science_do_zero2016} e\cite{ferrari2017} citam, como estratégia de treinamento, validação e teste as seguintes técnicas:
\begin{itemize}
	\item Resubstituição;
	\item Holdout;
	\item Validação cruzada;
	\item Bootstrap;
\end{itemize}

Na Resubstituição, segundo \cite{Boscarioli2017}, as medidas de avaliação dos classificadores são aplicadas no próprio conjunto de dados usados para indução do modelo. Essa técnica, embora tenha alguns vantagens discutidas em \cite{ferrari2017} e \cite{Boscarioli2017}, pode levar ao sobreajuste (\textit{overfitting}) discutido em \cite{data_science_do_zero2016}, \cite{aprenda_mineracao_fernando_amaral16} e \cite{Norvig2013}. Basicamente, o sobreajuste é quando se produz um modelo de bom desempenho com os dados de treinamento, mas que não lida bem com novos dados.

Na técnica de \textbf{Holdout}, pressupõem-se uma divisão, ou criação de dois subconjuntos de dados distintos, a partir do conjunto de dados disponível pra uso na indução do modelo/hipótese. Um desses subconjuntos será usado para treinamento (indução) do modelo de previsão e o segundo, para teste após o término do treinamento e, consequentemente, na aplicação das medidas de avaliação do modelo/hipótese.\cite{Boscarioli2017}

A imagem \ref{fig:img_holdout} mostra o funcionamento da técnica de holdout de forma mais detalhada
\begin{figure}[h!]
	\centering
	\includegraphics[width=.6\textwidth]{imagens/hold_out.png}
	\caption{Funcionamento da técnica holdout.}
	{\scriptsize Fonte:\cite{aprenda_mineracao_fernando_amaral16}}
	\label{fig:img_holdout}
\end{figure}

Na estratégia de validação cruzada, todos os dados farão parte, em algum momento, do conjunto de dados usado no teste do modelo/hipótese. A ideia é que cada exemplo sirva duplamente --- como dados de treinamento e dados de teste. Primeiro divide-se o conjunto em $k$ subconjuntos iguais. Em seguida realiza-se $k$ rodadas de aprendizagem; em cada iteração $\frac{1}{k}$ dos dados é retido como conjunto de teste e os exemplos restantes são usados como treinamento. Valores populares de $k$ são 5 e 10 --- o suficiente para uma estimativa estatisticamente provável que seja precisa a um custo 5-10 vezes maior no tempo de computação. Há também o extremo do $k = n$, também conhecido como \textbf{validação cruzada com omissão de um}. O método de validação cruzada permite que o modelo/hipótese seja avaliado uma série de vezes, cada série sendo conhecida como partição (ou \textit{fold}). Ao final, a avaliação pode ser realizada aplicando medidas estatísticas como média, desvio-padrão e intervalo de confiança ao conjunto de $k$ avaliações obtidas ou somando-se os desempenhos obtidos pelos $k$ modelos gerados e dividindo essa soma pelo número de exemplares original. \cite{Norvig2013}\cite{Boscarioli2017}\cite{ferrari2017} \cite{aprenda_mineracao_fernando_amaral16}

A imagem \ref{fig:img_cross_validation} mostra um exemplo didático de como funciona a validação cruzada:

\begin{figure}[h!]
	\centering
	\includegraphics[width=.6\textwidth]{imagens/cross_validation.png}
	\caption{Funcionamento da técnica \textit{cross validation}}
	{\scriptsize Fonte:\cite{fold_cross_validation:k-fold_nodate}}
	\label{fig:img_cross_validation}
\end{figure}

Já a técnica de \textit{Bootstrap} funciona de forma  parecida à estratégia \textit{holdout}. Ela também usa dois conjuntos, um de treinamento e outro para teste, porém durante o processo de formação dos subconjuntos, exemplares que já foram sorteados podem novamente serem contemplados, com probabilidade igual. É uma estratégia que permite, portanto, a reposição.

Neste trabalho, todos os algoritmos de classificação usados foram testados usando as técnicas de resubstituição, \textit{holdout} (com taxas de 70-30 e 60-40), além de \textit{cross-validation} com 3, 5 e 10 folds. Como explicado em \cite{WolpertMacready} e \cite{Wolpert:1996} não existe um algoritmo de aprendizado superior a todos os demais quando considerados todos os problemas de classificação possíveis (teorema \textbf{NFL}, ou \textit{No Free Lunch}), portanto, variações foram executadas nos experimentos em todas técnicas avaliadas, alterando-se os padrões para chegar a métricas e medidas de avaliação mais eficientes.

Há, conforme \cite{Boscarioli2017}, \cite{aprenda_mineracao_fernando_amaral16}, \cite{classification2013} e \cite{classification_survey2012} diversas medidas usadas na avaliação de classificadores. Uma delas (a que será usada neste trabalho) é a acurácia ou taxa de classificações corretas. A acurácia é dada, portanto, por:

\begin{equation}\label{acuracia}
\text{Acurácia} = |y-f(\aleph)=0|,
\end{equation}

em que $|\cdot|$ representa a contagem de vezes em que $\cdot$ é verdadeiro, $f$ é o modelo preditivo, $\aleph$ é o subconjunto de dados sob o qual o modelo está sendo avaliado, $f(\cdot)$ é a classificação fornecida pelo modelo preditivo para cada um dos exemplares (dos dados), e $y$ é a classe esperada como resposta. \cite{Boscarioli2017}

A acurácia de um classificador também pode ser descrita em termos do \textbf{erro de generalização} $\xi_g$, e uma função de perda binária e, portanto, ser interpretada como a probabilidade de ocorrer uma classificação correta. Dessa forma:

\begin{equation}
\text{Acurácia}_g=1 - \xi_g
\end{equation}

Ou seja, a acurácia é, basicamente o número de acertos (positivos) divido pelo número total de exemplos. Será a métrica mais usada para avaliar os classificadores neste trabalho.

\subsection{Redes Neurais artificiais}
As redes neurais instituem um campo da ciência da computação, parte da área da inteligência artificial, que busca efetivar modelos matemáticos que se assemelhem às redes neurais biológicas. Elas apresentam capacidade de adaptar seus parâmetros como resultado da interação com o meio externo. \cite{ferneda_redes_2006}\cite{Norvig2013}

De acordo com \cite[p. 47]{lima_ia_2016}, ``redes neurais podem ser caracterizadas como modelos computacionais com capacidades de adaptar, aprender, generalizar, agrupar ou organizar dados".

Inicialmente, portanto, se desenvolveram como uma estratégia de simular os processos mentais humanos, como reconhecimento de imagens e sons, e após, como instrumento tecnológico e eficienten para muitas tarefas. \cite{jin_development_2002}	

Para \cite{obaidat_multilayer_1994}, as redes neurais artificiais podem ser usadas efetivamente para prover soluções para um amplo espectro de aplicações, incluindo mapeamento de padrões e classificação, análise e codificação de imagens, processamento de sinais, otimização, manipulação de grafos, reconhecimento de caracteres, reconhecimento automático de alvo,  	fusão de dados, processamento de conhecimento, controle de qualidade, mercado de ações, processamento de hipotecas, triagem de créditos para empréstimos entre muitos outros problemas. 

Desde a década de 1940 com o trabalho de \cite{mcculloch_logical_1943} que se busca um modelo computacional que simule o cérebro humano e suas conexões. O interesse pela pesquisa nesta área cresceu e se desenvolveu durante os anos 50 e 60. É dessa época que \cite{rosenblatt_perceptron:_1958} sugeriu um método de aprendizagem para as redes neurais artificiais chamado \textit{percepton}. 

Até o final da década de 1960 muitos trabalhos foram feitos usando o percepton como modelo, mas ao final desta década, \cite{minsky_perceptrons:_1969} apresentaram significativas limitações do perceptron. A pesquisa diminui consideravelmente nos anos seguintes, porém durante  os  anos  80,  a excitação	ressurge mediante os avanços metodológicos importantes e, também, ao aumento dos recursos computacionais disponíveis. O  modelo  de  neurônio  artificial  da figura \ref{fig:neuronio} é uma
simplificação do apresentado por \cite[p. 36]{haykin_redes_2001}

\begin{figure}[h!]
	\centering
	\includegraphics[width=.7\textwidth]{imagens/modelo_matematico_neuronio.png}
	
	{\scriptsize Fonte:\cite[p. 36]{haykin_redes_2001}}
	\caption{Modelo matemático de um neurônio}
	\label{fig:neuronio}
\end{figure}

Este modelo acima (da figura \ref{fig:neuronio}) é composto por três elementos:

\begin{itemize}
	\item  um conjunto de $ n $ conexões de entrada ($ x_1, x_2, ... , x_n $), caracterizadas por pesos ($ p_1, p_2, ..., p_n $);
	\item um somador ($ \sum $) para acumular os sinais de entrada;
	\item uma função de ativação ($\varphi$) que, no caso do neurônio de McCullock-Pitts \cite{mcculloch_logical_1943} é uma função de limiar. \cite{ferneda_redes_2006} \cite{lima_ia_2016}
\end{itemize}

O comportamento das conexões entre os neurônios é simulado através de seus pesos  ($ p_1, p_2, ..., p_n $). Os valores podem ser positivos ou negativos (dependendo se a conexão é inibitiva ou excitativa. O efeito de um sinal proveniente de um neurônio é determinado pela multiplicação do valor do sinal recebido pelo peso da conexão correspondente ($x_i \times p_i$). Então é efetuada a soma dos valores $x_i \times p_i$ de todas as conexões e o valor resultante é enviado para a função de ativação que define a saída ($y$) do neurônio.\cite{Norvig2013}\cite{mcculloch_logical_1943}\cite{minsky_perceptrons:_1969}\cite{ferneda_redes_2006}\cite{haykin_redes_2001}

As redes neurais artificiais (\textbf{RNA}) se formam quando diversos neurônios se combinam. De forma resumida, ``uma rede neural artificial (RNA) pode ser vista como um grafo onde os nós são os neurônios e as ligações fazem a função das sinapses". Isto está demonstrado na figura \ref{fig:rna}

\begin{figure}[h!]
	\centering
	\includegraphics[width=.7\textwidth]{imagens/RNA.png}	
	\caption{Representação simplificada de uma RNA}
	{\scriptsize 	Fonte: \cite[p.26]{ferneda_redes_2006}}
	\label{fig:rna}
\end{figure}

As  redes  neurais  artificiais  se  diferem  pelas  suas arquiteturas e pela forma como os pesos associados às conexões são ajustados durante o processo de aprendizado.	A arquitetura de uma rede neural restringe o tipo de problema no qual a rede poderá ser utilizada, e é definida pelo  número  de  camadas  (camada única  ou múltiplas camadas), pelo número de nós em cada camada, pelo tipo de conexão entre os nós (\textit{feedforward} ou \textit{feedback}) e por sua topologia. \cite[p. 46-49]{haykin_redes_2001}

O desenvolvimento de uma rede neural artificial consiste em determinar sua arquitetura, ou seja, os números de camadas e de neurônios em cada camada, bem como o ajuste dos pesos na fase conhecida como treinamento.\cite{hagan_neural_1996} \cite{haykin_redes_2001}

Uma das características mais importantes de uma rede neural artificial é a habilidade de aprender através de exemplos e fazer inferências sobre o que aprendeu, melhorando, assim, o seu desempenho. As RNA's utilizam um algoritmo de aprendizagem que serve, basicamente, para ajustar os pesos de suas conexões. \cite{haykin_redes_2001} \cite{ferneda_redes_2006} \cite{lima_ia_2016} \cite{Norvig2013}. Aqui também há, cf. explicitado na seção \ref{aprendizagem_maquina}, duas formas básicas de aprendizado, o supervisionado e o não-supervisionado.

\subsection{SVM - Support Vector Machines}\label{SVM}
Segundo \cite{cortes_svm_1995}, o algoritmo SVM (\textit{Support Vector Machines}) é um dos mais efetivos para a tarefa de classificação.

Cf. \cite{goldschmidt2005},
\begin{quotation}
	No algoritmo SVM, o conjunto de dados de entrada é utilizado para construir uma \textit{função de decisão} $f(x)$, tal que:
	
	\begin{table}[h!]
		\centering
		\begin{tabular}{lll}			
			$ Se \ f(x_i) \geq 0,  $ & então   & $ y_i = 1 $  \\
			$ Se \ f(x_i) < 0, $ & então & $ y_i = -1 $  
		\end{tabular}
	\end{table}
	
	O algoritmo SVM constrói os denominados classificadores lineares, que separam o conjunto de dados por meio de um hiperplano que é a generalização do conceito de \textit{plano} para dimensões maiores que três.	
\end{quotation}

Assim, SVM, cf. \cite[p. 45]{aprenda_mineracao_fernando_amaral16} ``são um algoritmo de classificação que maximizam as margens entre instâncias mais próximas, dessa forma, é criado um vetor otimizado que é então utilizado para classificar novas instâncias".

Conforme se vê na figura \ref{fig:svm}, os dois vetores \textit{não pontilhados} são as margens otimizadas. As instâncias por onde as margens otimizadas passam são os vetores de suporte. O vetor pontilhado é a referência para classificar novas instâncias. Assim, a nova instância, na figura \ref{fig:svm} é classificada como triângulo.

\begin{figure}[h!]
	\centering
	\includegraphics[width=.7\textwidth]{imagens/vetores_de_suporte.png}	
	\caption{Vetores de Suporte}
	\label{fig:svm}
	{\scriptsize 	Fonte: \cite[p. 45]{aprenda_mineracao_fernando_amaral16}}
\end{figure}

Seguindo o estudo de \cite{mukkamala_intrusion_2002} há duas razões principais que levaram os autores do artigo citado de usarem SVMs para detecção de intrusão:o primeiro é a velocidade já que a performance é prioritariamente uma das características mais importantes para sistemas de detecção de intrusos. A segunda razão é a escalabilidade, pois, cf. os autores, SVMs são relativamente indiferentes ao número de \textit{data points} e a complexidade da classificação não depende da dimensionalidade do espaço de características. Dependendo da aplicação, ainda conforme os autores, uma vez que os dados estão classificados em duas classes, um algoritmo de otimização adequado pode ser usado, se necessário,  para identificação de mais características.

Neste trabalho, também foi usado, com boa eficácia (cf. se vê na seção \ref{resultados}) o algoritmo SVM.

\chapter{Experimentos/Resultados}\label{resultados}
Para os experimentos, um arquivo de políticas foi gerado a partir do proposto em \cite{sarkis2017} e, de acordo com o exposto na seção \ref{modelo_politica_utilizada}. O arquivo gerado possui cerca de 68 políticas nomeadas (constituindo a \textit{fase de seleção}\footnote{cf. seção \ref{mineracao_dados} deste trabalho.} da Mineração de Dados). 

Este arquivo foi usado nos testes preliminares da hipótese:  \textit{Converter a detecção de conflitos a um problema de classificação reestruturando os atributos} (colunas). Para este problema da detecção de conflitos diretos serão usadas técnicas de aprendizagem supervisionada. Para tanto, ao arquivo com as políticas, no pré-processamento foi acrescentada uma coluna rotulando os conflitos da seguinte forma: \textbf{1}: \textit{conflito direto} e \textbf{0}: \textit{sem conflito}.

A figura \ref{fig:aspecto_arquivo} demonstra o aspecto do arquivo das políticas geradas paraos experimentos deste trabalho. Na imagem, pode-se notar a classe (coluna) criada para guiar o aprendizado supervisionado dos algoritmos utilizados no estudo.

\begin{figure}[h!]
	\centering
	\includegraphics[width=.8\textwidth]{imagens/aspecto_arquivo_politicas.png}	
	\caption{Aspecto do arquivo das políticas geradas para os experimentos}
	\label{fig:aspecto_arquivo}
	{\scriptsize Fonte: compilação do autor}
\end{figure}

Dois \textbf{ambientes computacionais} foram utilizados para as tarefas de mineração: 
um \textbf{notebook}  Intel Core i5 vPro-8350U (8ª Geração de 64 bits com 1.70GHz e 8 GB de RAM, com SSD de 256 GB rodando Windows 10 Pro. 
O outro ambiente foi um \textbf{Desktop} Intel Core i7 vPro-6700 de 8ª geração de 64 bits com 3.40 Ghz e 20 GB de RAM, com HD de 1 TB rodando o Windows 10 Pro.

Ainda na fase de \textit{pré-processamento}, a coluna 9 (Conflito) foi transformada do tipo de dado \textit{Numérico para Nominal}. Para isso foi usada o softwate WEKA (descrito em \cite{eibe2016}) aplicado o filtro \textit{NumericToNominal} do software.

Logo após, mais de 30 experimentos foram realizados de forma preliminar no \textit{dataset} envolvendo os diversos algoritmos e muitos parâmetros alterados (a maioria com pequena ou nenhuma variação) para se chegar às técnicas finais que foram utilizadas nos posteriores experimentos e que serão explicitadas a seguir.

Utilizando-se a ferramenta WEKA (\cite{eibe2016}) para as últimas fases do KDD (Mineração de Dados), foram utilizados alguns algoritmos de classificação que segundo \cite{wu2007} são alguns dos mais utilizados na Mineração de Dados. Para avaliar o desempenho definiu-se o método cross-validation com 10 folds. Em seguida suas acurácias foram comparadas.

A tabela \ref{tab:acuracias} mostra o resultado destes experimentos:

\begin{table}[h!]
	\centering
	\caption{Acurácia dos classificadores}
	\label{tab:acuracias}
	\vspace{0.3cm}
	\begin{tabular}{p{6cm}c}
		\hline\\
		Classificador/Algoritmo& Acurácia  \\[10pt] 
		\hline
		Multi Layer Perceptron & 0.9705    \\
		Random Forest~         & 0.9558    \\
		J48                    & 0.9411    \\
		K* (K-star)            & 0.9411    \\
		Trees LMT              & 0.9117    \\
		IBk (KNN, com k =1)~   & 0.8970    \\
		JRip                   & 0.8970    \\
		SVM kernel linear~     & 0.8676    \\
		Nayve Bayes            & 0.8674    \\
		Random Tree            & 0.7794    \\
		\hline
	\end{tabular}
	\\[6pt]	\centering {\footnotesize Fonte: Elaborada pelo autor mediante experimentos}	
\end{table}

As figuras \ref{fig:saida_svm} e \ref{fig:saida_multilayerperceptron} mostram os resultados das classificações do arquivo de políticas usando, respectivamente, os classificadores/algoritmos: \textit{SVM} e o \textit{MultiLayer Perceptron} (que foram os principais citados nos trabalhos relacionados, cf. descrito na seção \ref{trabalhos_relacionados}). Importante ressaltar que outros classificadores, como o Random Forest, o J48, o K* e o KNN tiveram resultados superiores (em termos de acurácia e precisão) ao SVM, cf. mostrado na tabela \ref{tab:acuracias}. Entretanto, no referncial teórico, o SVM foi citado diversas vezes na detecção de alguns tipos de conflitos e em outras tarefas de classificação de diversos conjuntos de dados.
\begin{figure}[h!]
	\centering
	\includegraphics[width=.7\textwidth]{imagens/svm-resultados.png}	
	\caption{Saída do software WEKA. Classificador: SVM}
	\label{fig:saida_svm}
	{\scriptsize Fonte: compilação do autor}
\end{figure}
\begin{figure}[h!]
	\centering
	\includegraphics[width=.7\textwidth]{imagens/multilayerperceptron-resultados.png}
	\caption{Saída do software WEKA. Classificador: \textit{MultiLayer Perceptron}}
	\label{fig:saida_multilayerperceptron}
	{\scriptsize Fonte: compilação do autor}
\end{figure}

Assim, com uma acurácia de 97,05\% na classificação dos conflitos diretos, o algoritmo Multilayer Perceptron (que implementa uma rede neural sigmoide multicamadas) foi o que teve a maior acurácia, com 95,7\% de \textit{TP rate}(taxa de \textit{True Positives} ou verdadeiros positivos) para a classe 0 (não há conflito) e, somente, 4,3\% de \textit{FP rate}(taxa de Falsos Positivos) para a classe 1 (quando há conflito direto). Nos experimentos realizados (assim como se esperava inicialmente na hipótese deste trabalho --- baseado em evidências da literatura), este modelo algorítmico foi o mais eficiente para a detecção de conflitos diretos.

\chapter{Propostas para a dissertação}\label{propostas}
Para a pesquisa que resultará na dissertação de mestrado os seguintes pontos serão levantados, estudados e melhor definidos em termos dos objetivos do trabalho:
\begin{itemize}
	\item Pesquisar sobre o relacionamento entre entidades, ações e definições sobre políticas (para entendimento da propagação de políticas);
	\item Construção (teoria) e Programação (prática) do Perceptron (com \textit{backpropagation} para ajuste de pesos e atributos --- treinamento da rede neural);
	\item Análise da função de ativação no classificador;
	\item Análise teórica e construção da Função soma (e funções sigmóides de ativação do perceptron);
	\item Análise teórica e construção das múltiplas layers do perceptron;
	\item Usar Reinforcement Learning e Deep Learning para a detecção dos conflitos indiretos;
	\item Comparação com outros classificadores (preferencialmente, geométricos, como o KNN e o SVM, avaliando suas acurácias e eficiência.
\end{itemize}

\chapter{Conclusões}\label{conclusoes}
\begin{itemize}
	\item Esta pesquisa mostrou que é possível \textit{converter a detecção de conflitos a um problema de classificação} conforme demonstrado neste trabalho, especificamente, para os conflitos \textbf{\textit{diretos}};
	\item O classificador mais acurado, nos experimentos, foi, como se imaginava pela hipótese, o \textit{MultiLayer Perceptron} que é um classificador que usa \textit{backpropagation} para aprender usando perceptron de várias camadas para classificar instâncias desconhecidas \cite{eibe2016};
	\item Este será um dos classificadores usados para detectar conflitos indiretos. O outro será o SVM (e outros classificadores geométricos). Suas acurácias serão devidamente comparadas juntamente com a eficiência das soluções propostas.
\end{itemize}
%\chapter{Conteúdos específicos do modelo de trabalho %acadêmico}\label{cap_trabalho_academico}
%
%\section{Quadros}
%
%Este modelo vem com o ambiente \texttt{quadro} e impressão de Lista de quadros 
%configurados por padrão. Verifique um exemplo de utilização:
%
%\begin{quadro}[htb]
%\caption{\label{quadro_exemplo}Exemplo de quadro}
%\begin{tabular}{|c|c|c|c|}
%	\hline
%	\textbf{Pessoa} & \textbf{Idade} & \textbf{Peso} & \textbf{Altura} \\ \hline
%	Marcos & 26    & 68   & 178    \\ \hline
%	Ivone  & 22    & 57   & 162    \\ \hline
%	...    & ...   & ...  & ...    \\ \hline
%	Sueli  & 40    & 65   & 153    \\ \hline
%\end{tabular}
%\fonte{Autor.}
%\end{quadro}

%Este parágrafo apresenta como referenciar o quadro no texto, requisito
%obrigatório da ABNT. 
%Primeira opção, utilizando \texttt{autoref}: Ver o \autoref{quadro_exemplo}. 
%Segunda opção, utilizando  \texttt{ref}: Ver o Quadro \ref{quadro_exemplo}.

% ----------------------------------------------------------
% PARTE
% ----------------------------------------------------------
%\part{Referenciais teóricos}
% ----------------------------------------------------------

% ---
% Capitulo de revisão de literatura
% ---
%\chapter{Lorem ipsum dolor sit amet}
% ---

% ---
%\section{Aliquam vestibulum fringilla lorem}
% ---

%\lipsum[1]

%\lipsum[2-3]

% ----------------------------------------------------------
% PARTE
% ----------------------------------------------------------
%\part{Resultados}
% ----------------------------------------------------------

% ---
% primeiro capitulo de Resultados
% ---
%\chapter{Lectus lobortis condimentum}
% ---

% ---
%\section{Vestibulum ante ipsum primis in faucibus orci luctus et ultrices
%posuere cubilia Curae}
% ---

%\lipsum[21-22]

% ---
% segundo capitulo de Resultados
% ---
%\chapter{Nam sed tellus sit amet lectus urna ullamcorper tristique interdum
%elementum}
% ---

% ---
%\section{Pellentesque sit amet pede ac sem eleifend consectetuer}
% ---

%\lipsum[24]

% ----------------------------------------------------------
% Finaliza a parte no bookmark do PDF
% para que se inicie o bookmark na raiz
% e adiciona espaço de parte no Sumário
% ----------------------------------------------------------
\phantompart

% ---
% Conclusão
% ---
%\chapter{Conclusão}
% ---

%\lipsum[31-33]

% ----------------------------------------------------------
% ELEMENTOS PÓS-TEXTUAIS
% ----------------------------------------------------------
\postextual
% ----------------------------------------------------------

% ----------------------------------------------------------
% Referências bibliográficas
% ----------------------------------------------------------
\bibliography{bibliografia}

% ----------------------------------------------------------
% Glossário
% ----------------------------------------------------------
%
% Consulte o manual da classe abntex2 para orientações sobre o glossário.
%
%\glossary

% ----------------------------------------------------------
% Apêndices
% ----------------------------------------------------------

% ---
% Inicia os apêndices
% ---
%\begin{apendicesenv}

% Imprime uma página indicando o início dos apêndices
%\partapendices

% ----------------------------------------------------------
%\chapter{Quisque libero justo}
% ----------------------------------------------------------

%\lipsum[50]

% ----------------------------------------------------------
%\chapter{Nullam elementum urna vel imperdiet sodales elit ipsum pharetra ligula
%ac pretium ante justo a nulla curabitur tristique arcu eu metus}
% ----------------------------------------------------------
%\lipsum[55-57]

%\end{apendicesenv}
% ---


% ----------------------------------------------------------
% Anexos
% ----------------------------------------------------------

% ---
% Inicia os anexos
% ---
%\begin{anexosenv}

% Imprime uma página indicando o início dos anexos
%\partanexos

% ---
%\chapter{Morbi ultrices rutrum lorem.}
% ---
%\lipsum[30]

% ---
%\chapter{Cras non urna sed feugiat cum sociis natoque penatibus et magnis dis
%parturient montes nascetur ridiculus mus}
% ---

%\lipsum[31]

% ---
%\chapter{Fusce facilisis lacinia dui}
% ---

%\lipsum[32]

%\end{anexosenv}

%---------------------------------------------------------------------
% INDICE REMISSIVO
%---------------------------------------------------------------------
\phantompart
\printindex
%---------------------------------------------------------------------

\end{document}
