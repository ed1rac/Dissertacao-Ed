\chapter{Introdução} \label{introducao}
\section{Contextualização} \label{contextualizacao}
É fato que o volume de dados e informações cresce exponencialmente a cada ano \cite{alecrim2019}, portanto, há uma frequente e ininterrupta demanda por mais infraestrutura de TI nas empresas, nos governos e mesmo nos usuários domésticos \cite{machado2014} e, mais ainda, por um correto tratamento, destino e interpretação à imensidão de dados gerados por pessoas, empresas e governos. 

Em uma ampla variedade de campos, os dados estão sendo coletados e acumulados em um ritmo dramático.\cite{fayyad1996}\cite{lima_fraud_2012}. Há portanto esta crescente demanda por tratamento adequado a estes dados. Neste contexto se insere a mineração de dados com suas técnicas para tratamento e extração de conhecimento desse grande volume de dados.\cite{Boscarioli2017}\cite{ferrari2017}

Este trabalho usa diversas tarefas da mineração de dados para modelar uma hipótese que possibilite detectar conflitos em políticas de controle de acesso. Para isso, algoritmos de classificação serão explorados, descritos e utilizados.

As políticas de controle de acesso, como parte da segurança computacional, são uma das formas de garantir a proteção das informações dos sistemas.\cite{sarkis2017}

O controle de acesso, cf. \cite[p.1]{ueda_tese_2012} ``é um tema de pesquisa importante dentro do contexto de segurança de sistemas, pois é um dos componentes fundamentais em qualquer sistema de computação."

Segundo \cite{li_security_2006}, um aspecto muito relevante e muitas vezes tratado com pouca ênfase na construção de sistemas é a formulação de políticas de controle de acesso. 

Nas palavras de \cite[p.1]{ueda_tese_2012},
\begin{quotation}
	A definição dessas políticas é normalmente orientada por modelos que fornecem um conjunto de regras e mecanismos para o funcionamento seguro de uma representação abstrata de sistemas. Porém, a administração de tais políticas frequentemente \textit{se torna um processo complexo}, pois deve garantir que elas sejam eficientes e que não comprometam o desempenho dos sistemas. \textbf{\emph{[Grifo do autor.]}}
\end{quotation}

Uma política de controle de acesso descreve qual  ação um sujeito pode fazer (permissão), não pode fazer (proibição) ou é obrigado a fazer (obrigação) sobre um objeto em um dado contexto\cite{sarkis2017}

Muitas vezes as políticas de segurança apresentam conflitos. A detecção destes conflitos é o problema de pesquisa deste trabalho.



\section{Problema}\label{problema}
O problema investigado neste trabalho consiste em \textit{detectar conflitos diretos e indiretos usando técnicas de mineração de dados e aprendizagem de máquina}. É fato que, por exemplo, em \textit{logs} de controle de acesso, a quantidade de informações é grande e cresce exponencialmente conforme o uso. Sistemas grandes com múltiplos objetos e ações tem centenas, às vezes, milhares de políticas.\cite{fugini_information_2004}

\subsection{Hipótese}\label{hipótese}
Diante do contexto apresentado por \cite{fugini_information_2004} tem-se \textit{como hipótese deste trabalho}, portanto, que o problema de detectar conflitos em políticas pode ser convertido e transformado em uma tarefa de classificação da mineração de dados e que o uso de algoritmos de aprendizagem de máquina associados a técnicas de \textit{data mining} para detectar estes conflitos configure um método que apresente precisão e acurácia superiores a 95\%.

\subsection{Justificativa}\label{justificativa}
Lorem ipsum dolor sit amet, consectetur adipiscing elit. Duis rutrum maximus fermentum. Duis id quam nibh. Aliquam cursus eget mauris at fermentum. Nam iaculis posuere sapien. Praesent facilisis nisl ipsum, at gravida est scelerisque non. In sed luctus sem. Integer odio sem, feugiat eu eros eget, lacinia lobortis mi. Donec dapibus, tellus non consequat aliquet, dui massa suscipit dolor, nec pretium tortor massa vitae lorem. Quisque non faucibus tellus. Quisque dignissim nibh quis sem imperdiet bibendum. Duis nec dictum turpis. Vestibulum auctor leo nulla, sed venenatis purus placerat et. Nam euismod mauris in justo semper laoreet.

Cras rutrum faucibus eros et feugiat. In sit amet viverra lorem. Sed euismod purus eget sodales ultricies. Praesent et blandit lorem. Pellentesque ut tempus velit. Aenean et vulputate arcu. Suspendisse finibus, tellus eu sollicitudin rutrum, augue nisi maximus lacus, ut blandit ante mi ut nulla.

Aenean metus dui, rhoncus sit amet pretium ut, laoreet quis ipsum. Sed non tempus turpis, ac eleifend lacus. Integer vel dui finibus, imperdiet neque porttitor, venenatis nunc. Vivamus egestas, turpis quis porttitor tincidunt, purus risus suscipit felis, nec consequat justo quam at purus. Pellentesque consequat sem ut nibh malesuada pulvinar. Quisque lobortis pulvinar urna, vitae luctus nulla ultricies vestibulum. Sed accumsan tortor tellus, in aliquam tellus sollicitudin eu. Vestibulum ante ipsum primis in faucibus orci luctus et ultrices posuere cubilia curae; Duis justo ipsum, vestibulum at magna id, ultrices lobortis turpis. Vivamus porttitor scelerisque odio sit amet posuere. Morbi fermentum ultricies sem, dictum pulvinar turpis. 

\section{Objetivos}\label{objetivos}
\subsection{Objetivo geral}\label{objetivo_geral}
O objetivo deste trabalho é provar que o problema da detecção de conflitos em políticas pode ser convertido e resumido em um problema de \textit{data mining} (mineração de dados) resolvido pela tarefa da classificação além de modelar e resumir uma forma de detectar estes conflitos mediante o uso de diferentes algoritmos e técnicas da aprendizagem de máquina que consigam acurácias superiores a 95%.

\subsection{Objetivos específicos}
Os objetivos específicos deste trabalho são:
\begin{itemize}	
	\item Estabelecer a relação entre machine learning, técnicas de mineração de dados e o problema do conflito entre políticas;
	\item Determinar e comparar quais algoritmos e técnicas são mais adequados para cada tipo de conflito nas políticas (ou normas) usando as suas acurácias;
	\item Usar e comparar o desempenho, a precisão e taxa de acertos das principais técnicas usadas no aprendizado de máquina: redes neurais artificiais (RNA), Support Vector Machines (SVM) e redes neurais recorrentes e profundas.
	\item Reduzir o problema da detecção de conflitos à tarefa de classificação da mineração de dados;
	\item Usar frameworks de aprendizado de máquina como TensorFlow, Keras, Theano e Torch, na construção, treinamento e teste de redes neurais, comparando-os para o problema específico deste trabalho;
	\item Estabelecer a detecção de conflitos em políticas (ou normas) como uma  classe de problemas a serem resolvidos de forma eficiente por técnicas de aprendizagem de máquina.
\end{itemize}


\subsection{Solução Proposta/Hipóteses}\label{solucao_proposta}
Diante da hipótese apresentada em \ref{hipótese}, a solução para o problema apresentado neste trabalho em \ref{problema} resume-se a provar que \textit{converter} (ou \textit{reduzir}) a detecção de conflitos a um \textit{problema de classificação} da mineração de dados associado a técnicas de aprendizagem de máquina, reestruturando os atributos do \textit{dataset}, se necessário se configura um método com precisão superior a 95\%. 

\textit{A primeira solução/hipótese}, prioritariamente, para conflitos diretos é usar as técnicas e algoritmos de classificação (aprendizado supervisionado) para  realizar a detecção automatizada de conflitos. Para isso, propõem-se:
\begin{itemize}
	\item Como \textit{segunda solução/hipótese}: usar uma rede neural (um perceptron de uma camada) como técnica algorítmica para a detecção de conflitos e
	\item \textit{A terceira solução/hipótese} é usar uma rede neural multicamadas (com camadas ocultas), e retropropagação (backpropagation), comparando-a com outro classificador como o SVM para estabelecer qual técnica de detecção de conflitos em políticas é mais precisa.
	\item Realizado os experimentos, provar a hipótese mediante os resultados apresentados.
\end{itemize}

\subsection{Método de Pesquisa} 
O método de pesquisa relatando todos os passos necessários para demonstrar os objetivos descritos na seção \ref{objetivos} e de que forma eles foram atingidos serão pormenorizadamente detalhados na seção \ref{resultados}

\subsection{Resultados Esperados}\label{resultados_esperados}
Ao fim deste trabalho os seguintes resultados são esperados:
\begin{itemize}
	\item Provar que o problema da detecção de conflitos em políticas pode ser convertido e resumido em um problema de \textit{data mining} (mineração de dados) resolvido pela tarefa da classificação;
	\item Demonstrar e provar que o problema da detecção de conflitos é um prolema linearmente separável;
	\item Modelar e resumir uma forma de detectar estes conflitos mediante o uso de diferentes algoritmos e técnicas da aprendizagem de máquina que consigam acurácias superiores a 95\%;
	\item Estabelecer a relação entre machine learning, técnicas de mineração de dados e o problema do conflito entre políticas;
	\item Determinar e comparar quais algoritmos e técnicas são mais adequados para cada tipo de conflito nas políticas (ou normas) usando as suas acurácias;
	\item Usar e comparar o desempenho, a precisão e taxa de acertos das principais técnicas usadas no aprendizado de máquina: redes neurais artificiais (RNA), Support Vector Machines (SVM); redes neurais recorrentes e profundas;
	\item Estabelecer a detecção de conflitos em políticas (ou normas) como uma  classe de problemas a serem resolvidos de forma eficiente por técnicas de aprendizagem de máquina.
\end{itemize} 

\subsection{Limitações do Trabalho}\label{limitacoes}
Não faz parte do escopo deste trabalho:
\begin{itemize}
	\item Delinear um modelo de política com objetivos semânticos diferenciados. Para este trabalho será usado o modelo de políticas descrito em \cite{sarkis2017} e em \cite{sarkis:artigo:2016}
	\item Analisar comparativamente os modelos de extensão de políticas em um determinado contexto;
	\item Usar redes neurais convolucionais profundas na detecção dos conflitos;
	\item Abordar a semântica em políticas;	
\end{itemize} 