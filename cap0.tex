% --- -----------------------------------------------------------------
% --- Elementos usados na Capa e na Folha de Rosto.
% --- EXPRESSÔES ENTRE <> DEVERÂO SER COMPLETADAS COM A INFORMAÇÂO ESPECÍFICA DO TRABALHO
% --- E OS SÌMBOLOS <> DEVEM SER RETIRADOS 
% --- -----------------------------------------------------------------
\autor{EDKALLENN SILVA DE LIMA} % deve ser escrito em maiusculo

\titulo{Mineração de dados e aprendizagem de máquina na detecção de conflitos entre políticas de controle de acesso}

\instituicao{UNIVERSIDADE FEDERAL DO ACRE}

\orientador{DRA. LAURA COSTA SARKIS}

\local{RIO BRANCO - ACRE}

\data{2020} % ano da defesa

\comentario{Dissertação de Mestrado apresentada ao Programa de P\'{o}s-Gradua\c{c}\~{a}o em Computa\c{c}\~{a}o da \mbox{Universidade} Federal do Acre como requisito parcial para a obten\c{c}\~{a}o do Grau de \mbox{Mestre em Ciência da Computação}. \'{A}rea de concentra\c{c}\~{a}o: \mbox{Engenharia de Sistemas de Informação}} %preencha com a sua area de concentracao


% --- -----------------------------------------------------------------
% --- Capa. (Capa externa, aquela com as letrinhas douradas)(Obrigatorio)
% --- ----------------------------------------------------------------
\capa

% --- -----------------------------------------------------------------
% --- Folha de rosto. (Obrigatorio)
% --- ----------------------------------------------------------------

\pagestyle{ruledheader}
\setcounter{page}{1}
\pagenumbering{roman}

% --- -----------------------------------------------------------------
% --- Termo de aprovacao. (Obrigatorio)
% --- ----------------------------------------------------------------
\cleardoublepage
\thispagestyle{empty}

\vspace{-60mm}

\begin{center}
   {\large EDKALLENN SILVA DE LIMA}\\
   \vspace{7mm}

   Mineração de dados e aprendizagem de máquina na detecção de conflitos entre políticas de controle de acesso\\
  \vspace{10mm}
\end{center}

\noindent
\begin{flushright}
\begin{minipage}[t]{8cm}

Dissertação de Mestrado apresentada ao Programa de P\'{o}s-Gradua\c{c}\~{a}o em Computa\c{c}\~{a}o da Universidade Federal do Acre como requisito parcial para a obten\c{c}\~{a}o do \mbox{Grau} de Mestre em Ciência da Computação. \'{A}rea de concentra\c{c}\~{a}o: \mbox{Engenharia de Sistemas de Informação} %preencha com a sua area de concentracao

\end{minipage}
\end{flushright}
\vspace{1.0 cm}
\noindent
Aprovada em <MES> de 2020. \\
\begin{flushright}
  \parbox{11cm}
  {
  \begin{center}
  BANCA EXAMINADORA \\
  \vspace{6mm}
  \rule{11cm}{.1mm} \\
    Prof. DRA. LAURA COSTA SARKIS - Orientador, UFAC \\
    \vspace{6mm}
  \rule{11cm}{.1mm} \\
    Prof. <NOME DO AVALIADOR>, <INSTITUI\c{C}\~AO>\\
    \vspace{6mm}
  \rule{11cm}{.1mm} \\
    Prof. <NOME DO AVALIADOR>, <INSTITUI\c{C}\~AO>\\
  \vspace{4mm}
  \rule{11cm}{.1mm} \\
    Prof. <NOME DO AVALIADOR>, <INSTITUI\c{C}\~AO>\\
    \vspace{6mm}

  \end{center}
  }
\end{flushright}
\begin{center}
  \vspace{4mm}
  RIO BRANCO - ACRE \\
  %\vspace{6mm}
  2020

\end{center}

% --- -----------------------------------------------------------------
% --- Dedicatoria.(Opcional)
% --- -----------------------------------------------------------------
\cleardoublepage
\thispagestyle{empty}
\vspace*{200mm}

\begin{flushright}
{\em 
Dedicat\'oria(s): Este trabalho não seria possível sem a educação que me foi concedida, os ensinamentos e a sabedoria oriundos de você, Lucimar do Rego Albuquerque de Lima, muito mais que uma mãe, uma inspiração. <IN MEMORIAN>
}
\end{flushright}
\newpage


% --- -----------------------------------------------------------------
% --- Citação.(Opcional)
% --- -----------------------------------------------------------------
\cleardoublepage
\thispagestyle{empty}
\vspace*{200mm}

\begin{flushright}
	{\em 
		“Assim como casas são feitas de pedras, a ciência é feita de fatos. Mas uma pilha de pedras não é uma casa e uma coleção de fatos não é, necessariamente, ciência”.  \\ \textbf{Jules Henri Poincaré}, matemático, físico e filósofo da ciência francês
	}
\end{flushright}
\newpage




% --- -----------------------------------------------------------------
% --- Agradecimentos.(Opcional)
% --- -----------------------------------------------------------------
\pretextualchapter{Agradecimentos}
\hspace{5mm}
Lorem ipsum dolor sit amet, consectetur adipiscing elit. Duis rutrum maximus fermentum. Duis id quam nibh. Aliquam cursus eget mauris at fermentum. Nam iaculis posuere sapien. Praesent facilisis nisl ipsum, at gravida est scelerisque non. In sed luctus sem. Integer odio sem, feugiat eu eros eget, lacinia lobortis mi. Donec dapibus, tellus non consequat aliquet, dui massa suscipit dolor, nec pretium tortor massa vitae lorem. Quisque non faucibus tellus. Quisque dignissim nibh quis sem imperdiet bibendum. Duis nec dictum turpis. Vestibulum auctor leo nulla, sed venenatis purus placerat et. Nam euismod mauris in justo semper laoreet.

Cras rutrum faucibus eros et feugiat. In sit amet viverra lorem. Sed euismod purus eget sodales ultricies. Praesent et blandit lorem. Pellentesque ut tempus velit. Aenean et vulputate arcu. Suspendisse finibus, tellus eu sollicitudin rutrum, augue nisi maximus lacus, ut blandit ante mi ut nulla.

Aenean metus dui, rhoncus sit amet pretium ut, laoreet quis ipsum. Sed non tempus turpis, ac eleifend lacus. Integer vel dui finibus, imperdiet neque porttitor, venenatis nunc. Vivamus egestas, turpis quis porttitor tincidunt, purus risus suscipit felis, nec consequat justo quam at purus. Pellentesque consequat sem ut nibh malesuada pulvinar. Quisque lobortis pulvinar urna, vitae luctus nulla ultricies vestibulum. Sed accumsan tortor tellus, in aliquam tellus sollicitudin eu. Vestibulum ante ipsum primis in faucibus orci luctus et ultrices posuere cubilia curae; Duis justo ipsum, vestibulum at magna id, ultrices lobortis turpis. Vivamus porttitor scelerisque odio sit amet posuere. Morbi fermentum ultricies sem, dictum pulvinar turpis. 

% --- -----------------------------------------------------------------
% --- Resumo em portugues.(Obrigatorio)
% --- -----------------------------------------------------------------
\begin{resumo}

  Um dos principais componentes de segurança em sistemas computacionais é o controle de acesso, que é um mecanismo que verifica todas as solicitações de dados e recursos gerenciados pelos sistemas, determinando quando as solicitações são permitidas ou negadas. Uma política de controle de acesso descreve qual  ação um sujeito pode fazer (permissão), não pode fazer (proibição) ou é obrigado a fazer (obrigação) sobre um objeto em um dado contexto. Muitas vezes as políticas de segurança apresentam conflitos. Esses conflitos podem gerar problemas, como acessos não autorizados ou negações a acessos legítimos. Diante disto, a detecção de conflitos entre políticas de controle de acesso torna-se uma atividade sensível e crítica dentro de sistemas. Este trabalho estuda se é pertinente e válido o uso de mineração de dados associada a técnicas de aprendizagem de máquina para detectar os conflitos entre estas diversas políticas. Seu objetivo é modelar uma forma de detecção automática, mediante o uso de técnicas de mineração de dados (e \textit{aprendizagem de máquina}) dos conflitos, sejam eles, diretos ou indiretos entre as políticas de controle de acesso de um sistema.

{\hspace{-8mm} \bf{Palavras-chave}}: Controle de Acesso. Mineração de dados. Aprendizagem de máquina. Conflitos diretos. Conflitos indiretos. Detecção de conflitos.

\end{resumo}

% --- -----------------------------------------------------------------
% --- Resumo em lingua estrangeira.(Obrigatorio)
% --- -----------------------------------------------------------------
\begin{abstract}

  One of the main security components in computer systems is access control, which is a mechanism that checks all requests for data and resources managed by the systems, determining when requests are allowed or denied. An access control policy describes what action a subject can do (permission), cannot do (prohibition) or is obliged to do (obligation) on an object in a given context. Security policies often have conflicts. These conflicts can lead to problems, such as unauthorized access or denial of legitimate access. Given this, the detection of conflicts between access control policies becomes a sensitive and critical activity within systems. This paper studies whether the use of data mining associated with machine learning techniques to detect conflicts between these different policies is relevant and valid. Its objective is to model a form of automatic detection, using data mining techniques (and machine learning) of conflicts, whether direct or indirect between a system's access control policies.

{\hspace{-8mm} \bf{Keywords}}: Access control. Data mining. Machine learning. Direct conflicts. Indirect conflicts. Conflict detection.

\end{abstract}

% --- -----------------------------------------------------------------
% --- Lista de figuras.(Opcional)
% --- -----------------------------------------------------------------
%\cleardoublepage
\listoffigures


% --- -----------------------------------------------------------------
% --- Lista de tabelas.(Opcional)
% --- -----------------------------------------------------------------
\cleardoublepage
%\label{pag:last_page_introduction}
\listoftables
\cleardoublepage

% --- -----------------------------------------------------------------
% --- Lista de abreviatura.(Opcional)
%Elemento opcional, que consiste na relação alfabética das abreviaturas e siglas utilizadas no texto, seguidas das %palavras ou expressões correspondentes grafadas por extenso. Recomenda-se a elaboração de lista própria para cada %tipo (ABNT, 2005).
% --- ----------------------------------------------------------------
\cleardoublepage
\pretextualchapter{Lista de Abreviaturas e Siglas}
\begin{tabular}{lcl}
<ABREVIATURA> & : & <SIGNIFICADO>;\\
<ABREVIATURA> & : & <SIGNIFICADO>;\\
<ABREVIATURA> & : & <SIGNIFICADO>;\\
\end{tabular}
% --- -----------------------------------------------------------------
% --- Sumario.(Obrigatorio)
% --- -----------------------------------------------------------------
\pagestyle{ruledheader}
\tableofcontents


